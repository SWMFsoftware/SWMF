
\chapter{Introduction}

This document descibes the installation, configuration, execution and 
usage of the Comprehensive Inner Magnetosphere Ionosphere (CIMI) model.

\section{Acknowledgments}


\section{The CIMI model in a Few Paragraphs}


\section{System Requirements}

In order to install and run CIMI the following minimum system
requirements apply.

\begin{itemize}
\item The CIMI model runs only under the UNIX/Linux operating systems.  This now
  includes Macintosh system 10.x because it is based on BSD UNIX.  The
  CIMI model does not run under any Microsoft Windows operating system.
\item A FORTRAN 90 or greater compiler must be installed.
\item The Perl interpreter must be installed.
\item A version of the Message Passing Interface (MPI) library must be
  installed.
\item Small tests such as single  field-line simulations can be done on a 
  single processor
\item Very large runs require many more processors.
\item In order to generate the documentation you must have LaTex installed on
your system.  The PDF generation requires the {\tt dvips} and {\tt ps2pdf}
utilities. 

\end{itemize}


In addition to the above requirements, the CIMI output is designed to
be visualized using IDL.  You may be able to
visualize the output with other packages, but formats and scripts have
been designed for only this visualization software.


%-----------------------------------------------------------------------
% Chapter 2
%-----------------------------------------------------------------------

\chapter{Quick Start}

\section{A Brief Description of the CIMI Distribution}

The top level directory contains the following subdirectories:
\begin{itemize}
\item {\tt src}       - Source code of the main code
\item {\tt srcSAMI3}  - Source code of SAMI3 model
\item {\tt Scripts}   - shell and Perl scripts
\item {\tt bin}       - location of executable
\item {\tt doc}       - the documentation directory 
\item {\tt share}     - shared scripts and source code
\item {\tt util}      - general utilities such as TIMING and NOMPI
\end{itemize}
and the following files
\begin{itemize}
\item {\tt README}        - a short instruction on installation and usage
\item {\tt Makefile}      - the main makefile
\item {\tt Config.pl}     - Perl script for (un)installation and configuration
\end{itemize}

\section{General Hints}

\subsubsection{Getting help with scripts and the Makefile}

Most of the Perl and shell scripts that are distributed with the PWOM
provide help which can be accessed as follows using the {\tt -h} flag.
For example, 
\begin{verbatim}
  Config.pl -h
\end{verbatim}
will provide a detailed listing of the options and capabilities of the
{\tt Config.pl} script.  In addition, you can find all the possible
targets  that can be built by typing
\begin{verbatim}
make help
\end{verbatim}

\subsubsection{Input commands: PARAM.XML}
A very useful set of files to become familiar with are the {\tt PARAM.XML}
files.

This file contains a complete list of all input commands for the
component as well as the allowed ranges for each of the input parameters.
Although the XML format makes the files a little hard to read, they are
extremely useful.  A typical usage is to cut and paste commands out of the
PARAM.XML file into the PARAM.in file for a run.


\section{Installing the Code}


The CIMI model needs to know what architecture you are running the code on
and what FORTRAN compiler will be used.  For most platforms and compilers,
it can figure this out all by itself. To install PWOM run the command:
\begin{verbatim}
  Config.pl -install
\end{verbatim}
in the main directory. This creates {\tt Makefile.def} with
the correct absolute path to the base directory and {\tt Makefile.conf}
which contains the operating system and compiler specific part of
the Makefile. If the compiler is not the default one for a given
platform (e.g. not the NAG f95 compiler for a Linux platform) then
the compiler must be specified explicitly with the {\tt -compiler}
flag. If the MPI header file is not the default one, it can be
specified with the {\tt -mpi} flag. For example on a Mac machine
one can select the g95 compiler with the Intel version of the 
MPI library using
\begin{verbatim}
  Config.pl -install -compiler=g95 -mpi=Intel
\end{verbatim}
To uninstall CIMI type
\begin{verbatim}
  Config.pl -uninstall
\end{verbatim}
If the uninstallation fails (this can happen if some makefiles are missing)
do reinstallation with
\begin{verbatim}
  Config.pl -install
\end{verbatim}
and then try uninstalling the code again.
To get a complete description of the {\tt Config.pl}  script type
\begin{verbatim}
  Config.pl -h
\end{verbatim}

\section{Creating Documentation}

The documentation for PWOM can be generated from the distribution by
the command
\begin{verbatim}
  make PDF
\end{verbatim}
which creates the user manual
\begin{verbatim}
  doc/PWOM.pdf
\end{verbatim}
In order for this to work you must have
LaTex installed on your system (and dvips and ps2pdf).  

To clean the intermediate files type
\begin{verbatim}
  cd doc/Tex
  make clean
\end{verbatim}
To remove all the created documentation type
\begin{verbatim}
  cd doc/Tex
  make cleanall
\end{verbatim}

\section{Building and Running an Executable}


Compilation flags, such as the precision and optimization 
level are stored in {\tt Makefile.conf}. This file is created on
installation of CIMI and has defaults which are appropriate for
your system architecture.  The precision of reals
can be changed to single precision (for example) by typing
\begin{verbatim}
  Config.pl -single
\end{verbatim}
while the compiler flags can be modified with
\begin{verbatim}
  Config.pl -debug -O0
\end{verbatim}
to debug the code with 0 optimization level, and
\begin{verbatim}
  Config.pl -nodebug -O4
\end{verbatim}
to run the code at maximum optimization level and without the debugging flags.

To build the executable {\bf bin/CIMI.exe}, type:
\begin{verbatim}
  make
\end{verbatim} 
Depending on the configuration, the compiler settings and the machine 
that you are compiling on, this can take from 1 to up to 5 minutes.  

The {\tt CIMI.exe} executable should be run in a sub-directory, since a 
large number of files are created in each run.  
To create this directory use thecommand:
\begin{verbatim}
  make rundir
\end{verbatim} 
This command creates a directory called {\tt run}.  You can either
leave this directory as named, or {\tt mv} it to a different name.  It
is best to leave it in the same CIMI directory, since
keeping track of the code version associated with each run is quite
important.  The {\tt run} directory will contain links to the codes
which were created in the previous step as well as subdirectories
where input and output will reside.

Here we assume that the {\tt run} directory is still called {\tt
run}:
\begin{verbatim}
  cd run
\end{verbatim}
In order to run CIMI you must have the input files: 
PARAM.in. The PARAM.in 
file contains the detailed commands for controlling what you want the
code to do during the run.  The default PARAM.in 
file in the run directory  suitable to perform a simple test.

To run CIMI interactively on four processors:
\begin{verbatim}
cd run
mpirun -np 4 CIMI.exe
\end{verbatim}

To recompile the executable with different compiler settings you have
to use the command
\begin{verbatim}
make clean
\end{verbatim}
before recompiling the executables. It is possible to recompile
only a component or just one subdirectory if the {\tt make clean}
command is issued in the appropriate directory.

\section{Restarting a Run}

There are several reasons for restarting a run. A run may fail
due to a run time error, due to hardware failure, due to 
software failure (e.g. the machine crashes) or because the
queue limits are exceeded. In such a case the run can be continued from
the last saved state of CIMI. 

The restart files are saved at the frequency determined in the PARAM.in file.
Normally the restart files are saved into the IM/restartOUT directory. 
The files in that directory need only be copied into the PW/restartIN 
directory inorder to prepare a restarted run.

\section{What next?}

Hopefully this section has guided you through installing CIMI and
given you a basic knowledge of how to run it.  So what next?

We suggest that you read all of chapter \ref{chapter:basics}, which
outlines the basic features of CIMI as well as some things you
really must know in order to use the code.  Once you have done this you
are ready to experiment.  Chapter \ref{chapter:examples} gives several 
examples which are intended to make you familiar with the use of CIMI.  We suggest that you try them!

%\end{document}
