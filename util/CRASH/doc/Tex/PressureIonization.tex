\section{Pressure ionization}
We account for pressure ionization via energy levels correction as follows:
\begin{equation}
\Delta E_{i,n,l} = f(i,n,l) (r_{iono}[a])^{-\theta}, \qquad
E_{i,n,l} = E^*_i + E^{exc}_{i,n} + \Delta E_{i,n,l},
\end{equation}
where the function $f$ and the constant value $\theta$ must be adjusted
according to the particular model being used. Now we distingish quantum states also by
the azimuthal quantum number, $l$, because the effect of pressure ionization
depends on the orbital type.

Besides the continuum lowering due to Coulomb interactions, the effect of
pressure ionization raises the excitation levels. When the excitation energy
exceeds the ionization energy, the corresponding excitation level vanishes.
Application of the Madelung theory results in the condition for
energy level elimination as follows:
\begin{equation}
E^{exc}_{i,n} + \Delta E_{i,n,l} > I_i - 3 (2i+1) TL.
\end{equation}

We can still use the
Eqs.(\ref{generalE}), (\ref{generalP}), (\ref{generalCv}),
(\ref{generalPT}), (\ref{generalCompr}), where mean values and
covariance should now be calculated over all possible triplets $(i, n, l)$.
Substitution of the partial devivatives of $E$ with respect to $V$ as follows:
\begin{eqnarray}
-V \frac{\partial (\Delta E_{i,n,l})}{\partial V} =&
\frac{\theta}{3} f(i,n,l) (r_{iono}[a])^{-\theta} &=
\frac{\theta}{3} \Delta E_{i,n,l}, \\
V^2 \frac{\partial^2 (\Delta E)}{\partial V^2} =&
\frac{\theta (\theta + 3)}{9} f(i,n,l) (r_{iono}[a])^{-\theta} &=
\frac{\theta (\theta + 3)}{9} \Delta E_{i,n,l},
\end{eqnarray}
gives us the following formulas for thermodynamic derivatives:
\begin{equation}
P = n_a \left[ T(1 + ZR^+ - L \langle i^2 \rangle) +
\frac{\theta}{3} \langle \Delta E \rangle \right],
\end{equation}
\begin{equation}
{\cal E} = n_a \left[ \frac32 T (1 + ZR^+) + \langle E \rangle \right],
\end{equation}
\begin{equation}
C_V = n_a \left[ \frac32 + \frac{\langle \delta^2 E \rangle}{T^2} +
\frac{15}{4} ZR^+ - \frac{(\frac32 Z - \frac{\langle \delta i \delta E \rangle}{T})^2}
{\langle \delta^2 i \rangle + ZR^-} \right],
\end{equation}
\begin{eqnarray}
\nonumber \frac{\partial P}{\partial T} &=& n_a \Biggl[ 1 + \frac52 ZR^+ -
 \left( Z + L \langle \delta i \delta (i^2) \rangle
- \frac{\theta}{3} \frac{1}{T} \langle \delta i \delta \Delta E \rangle \right)
\frac{\frac32 Z - \frac{\langle \delta i \delta E \rangle}{T}}
{\langle \delta^2 i \rangle + ZR^-} - \\
&-& L \frac{\langle \delta E \delta (i^2)}{T} +
\frac{\theta}{3} \frac{\langle \delta E \delta \Delta E \rangle}{T^2} \Biggr],
\end{eqnarray}
\begin{eqnarray}
\nonumber V \frac{\partial P}{\partial V} &=& n_a T \Biggl[ -1 -
\frac{(Z + \langle Li^2 - \frac{\theta}{3} \frac{\Delta E}{T} \rangle)^2}
{\langle \delta^2 i \rangle + ZR^-} +
\frac43 L \langle i^2 \rangle -
\frac{\theta (\theta+3)}{9} \frac{\langle \Delta E \rangle}{T} + \\
&+& \left\langle \delta^2 \left( Li^2 - \frac{\theta}{3} \frac{\Delta E}{T} \right) \right\rangle \Biggr].
\end{eqnarray}
