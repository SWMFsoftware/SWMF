% LyX 1.5.5 created this file.  For more info, see http://www.lyx.org/.
% Do not edit unless you really know what you are doing.
\documentclass[english,12pt]{revtex4}
\usepackage[T1]{fontenc}
\usepackage[latin9]{inputenc}
\usepackage{nicefrac}
\usepackage{graphicx}
\usepackage{multirow}
\usepackage{array}

\makeatletter

%%%%%%%%%%%%%%%%%%%%%%%%%%%%%% LyX specific LaTeX commands.
%% Because html converters don't know tabularnewline
\providecommand{\tabularnewline}{\\}

%%%%%%%%%%%%%%%%%%%%%%%%%%%%%% User specified LaTeX commands.
%% LyX 1.5.5 created this file.  For more info, see http://www.lyx.org/.
%% Do not edit unless you really know what you are doing.



\usepackage{nicefrac}


\makeatletter

%%%%%%%%%%%%%%%%%%%%%%%%%%%%%% LyX specific LaTeX commands.
%% Because html converters don't know tabularnewline


%%%%%%%%%%%%%%%%%%%%%%%%%%%%%% User specified LaTeX commands.
%% LyX 1.5.5 created this file.  For more info, see http://www.lyx.org/.
%% Do not edit unless you really know what you are doing.



\usepackage{nicefrac}



\makeatletter

%%%%%%%%%%%%%%%%%%%%%%%%%%%%%% LyX specific LaTeX commands.
%% Because html converters don't know tabularnewline


%%%%%%%%%%%%%%%%%%%%%%%%%%%%%% User specified LaTeX commands.
%% LyX 1.5.5 created this file.  For more info, see http://www.lyx.org/.
%% Do not edit unless you really know what you are doing.



\usepackage{nicefrac}

\begin{document}
\section{Multi-group diffusion}
{\bf Governing equation} to describe the radiation transport in the multigrop diffusion 
approximation may be written as the partial differential equation for the spectral 
energy density, $E_{\varepsilon}$, which is related to the unit ov volume and the unit
interval of the photon energy, $\varepsilon$. The energy is assumed to be integrated 
over the solid angle of directions of the photon propagation. Once  the spectral energy
is integrated 
density over photon energies, the total radiation energy density is obtained:
\begin{equation}\label{eq:mg0}
E=\int_0^\infty{E_\varepsilon d\varepsilon}.
\end{equation}

The governing equation for the spectral energy density is as follows:
\begin{equation}\label{eq:mg1}
\frac{\partial E_\varepsilon}{\partial t}+\nabla\cdot({\bf u}E_\varepsilon)-(\gamma_R-1)(\nabla\cdot{\bf u})\varepsilon\frac{\partial E_\varepsilon}{\partial \varepsilon}=
{\rm diffusion + emission - absorption}.
\end{equation}
The second and third terms in the left hand side of Eq.(\ref{eq:mg1}) express the time evolution of the spectral energy density resulting from: (1) the radiation 
advection and comression with the background, which moves with the velocity, ${\bf u}$; as well as (2) the photon systematic blue (red) shift in the convergent (divergent) 
motions, which is analogous to the first order Fermi acceleration of chanrged particles in a moving plasma with the frozen in magnetic field. 
Herwith $\gamma_R=4/3$ is the adiabat index of a relativistic (photon) gas. The processes described by the symbolic terms in the right hand side of Eq.(\ref{eq:mg1})
are described below.

{\bf A set of multi-group equations} may be indroduced when we choose a set of frequency groups. Here we enumerate groups with the index, $g=1, G$. The interval of the 
photon energies, relating to the $g$th group is denoted as $[\varepsilon_{g-1/2},\varepsilon_{g+1/2}]$. The discrete set of unknowns, $E_g$, is introduced in terms of the
integrals of the spectral energy density of the frequency group interval:
\begin{equation}\label{eq:mg2}
E_g=\int_{\varepsilon_{g-1/2}}^{\varepsilon_{g+1/2}}{E_\varepsilon d\varepsilon}. 
\end{equation}
Note that according to Eqs.(\ref{eq:mg0},\ref{eq:mg2})
\begin{equation}
E=\sum_g{E_g}.
\end{equation}

Although some of the formulae below are not sensitive to the choice of the group set, here we specify the boundaries of the frequency groups to be such that 
{\it the frequency logarithm} is equally spaced (rather than the frequency itself):
\begin{equation}\label{eq:mg3}
\log(\varepsilon_{g+1/2}) - \log(\varepsilon_{g-1/2}) = \Delta(\log \varepsilon) = const.
\end{equation} 
Note that as long as the number of groups, $G$, tends to infinity, the ratio $E_g/\Delta(\log \varepsilon)$ tends to the local value of 
$\varepsilon E_\varepsilon$, rather than to $E_\varepsilon$. Therefore, the grequency integrals on the equally spaced logarithmic frequency 
grid allow us to approximate not
a spectral energy density, but its product by the photon energy:
\begin{equation}\label{eq:mg4}
\frac{E_g}{ \Delta(\log \varepsilon)}\approx \varepsilon E_\varepsilon.
\end{equation}

Now we can integrate Eq.(\ref{eq:mg1}) to arrive at the desired set of the multigroup equations:
\begin{eqnarray}
\frac{\partial E_q}
{\partial t}&+&
\nabla\cdot({\bf u}E_g)+
(\gamma_R-1)(\nabla\cdot{\bf u})E_g+\nonumber\\
&+&\left[-(\gamma_R-1)(\nabla\cdot{\bf u})\right]\left[
\varepsilon_{g+1/2} E_\varepsilon
(\varepsilon_{g+1/2})-
\varepsilon_{g-1/2} E_\varepsilon(\varepsilon_{g-1/2})\right]
=\nonumber\\
&=&
\int_{\varepsilon_{g-1/2}}^{\varepsilon_{g+1/2}}
{
{\rm (diffusion + emission - absorption)}
d\varepsilon}.\label{eq:mg5}
\end{eqnarray}
If the number of frequency groups, $G$ is sufficiently large, we employ the approximation as in Eq.(\ref{eq:mg4}), which allows us to close Eq.(\ref{eq:mg5}) in the 
following form:
\begin{eqnarray}
\frac{\partial E_q}
{\partial t}&+&
\nabla\cdot({\bf u}E_g)+
(\gamma_R-1)(\nabla\cdot{\bf u})E_g+
%\nonumber\\&+&
\frac{-(\gamma_R-1)(\nabla\cdot{\bf u})}{ \Delta(\log \varepsilon)}\left[
E_{g+1/2}- E_{g-1/2})\right]
=\nonumber\\
&=&
\int_{\varepsilon_{g-1/2}}^{\varepsilon_{g+1/2}}
{
{\rm (diffusion + emission - absorption)}
d\varepsilon}.\label{eq:mg6}
\end{eqnarray}
where the values $E_{g\pm1/2}$ should be interpolated from the mesh-centered values $E_g$ towards the frequency values corresponding to the inter-group boundary.

Note that we arrived to difference-differential equation, with the left-hand side including: (1) the conservative advection of the radiation energy density with
the velocity ${\bf u}$; the work done by the radiation pressure $P_g=(\gamma_R-1)E_g$; (3) and, as a new element, a linear conservative advection with respect to
the log-frequency coordinate. The flux-to-control-volume ratio for the latter effect equals 
$$F_{g-1/2}=-(\gamma_R-1)(\nabla\cdot{\bf u})E_{g-1/2}/ \Delta(\log \varepsilon).$$ 
Eq.(\ref{eq:mg6}) provides us a ready numerical scheme to solve this advection numerically. The recipi to construct a numerical flux for a linear advection equation
are well-known. Specifically, we apply a limited reconstruction procedure, with the 
'superbee' limiter function,
$$L(a,b)=\frac12[{\rm sign}(a)+{\rm sign}(b)]\min(\max(|a|,|b|),2|a|,2|b|),$$ to obtain left and right interpolated values, 
$$E^{(L)}_{g-1/2}=E_{g-1}+\frac12L(E_g-E_{g-1},E_{g-1}-E_{g-2}),$$
$$ E^{(R)}_{g-1/2}=E_g-\frac12L(E_{g+1}-E_{g},E_{g}-E_{g-1})$$. 
With these interpolated values  the upwinded numerical flux is constructed as 
follows:
\begin{eqnarray}
F_{g-1/2}&=&-(\gamma_R-1)(\nabla\cdot{\bf u})E^{(R)}_{g-1/2}/ \Delta(\log \varepsilon),\qquad (\nabla\cdot{\bf u})\ge0,\nonumber\\
F_{g-1/2}&=&-(\gamma_R-1)(\nabla\cdot{\bf u})E^{(L)}_{g-1/2}/ \Delta(\log \varepsilon),\qquad (\nabla\cdot{\bf u})\le0.\label{eq:mg7}
\end{eqnarray}
Eqs.(\ref{eq:mg6}) with the numerical fluxes as in Eq.(\ref{eq:mg7}) may be further 
discretized using the control volume methods and solved numerically, assuming the 
right hand side to be zero, as the coupled element of the system of hydrodynamic 
(or magneto-hydro-dynamic) equations involving the ratiation pressure contributions. 
Note, that suming up Eqs.(\ref{eq:mg6}) we obtain the equation of the gray radiation
diffusion:
\begin{equation}\label{eq:mg8}
\frac{\partial E}{\partial t}+\nabla\cdot({\bf u}E)+(\gamma_R-1)(\nabla\cdot{\bf u}) E=\int_0^\infty{
{(\rm diffusion + emission - absorption)}d\varepsilon}.
\end{equation}
If, again, we assume the right hand side of Eq.(\ref{eq:mg8}) to be zero, then only Eq.(\ref{eq:mg8}) is two-way coupled to the system of equation of the plasma motion,
since in these equations the radiation effects the plasma motion only via the total radiation pressure:
\begin{equation}
P=(\gamma_R-1)E.
\end{equation}
On advancing the solution of this coupled system through the time step, the numerical solution of each of Eqs.(\ref{eq:mg6}) may be also advanced through the time step, 
$\Delta t$, as 
long as ${\bf u}$ is known. Depending on the value of $\Delta(\log \varepsilon)$ one may prefer to treat the advection over frequency as: (1) an extra flux (the control
volume in this case is treated as the four-dimentional rectangular box $\Delta x*\Delta y*\Delta z*\Delta (\log \varepsilon)$); or (2) an extra advance operator, which can be
split out and hanled separately. In the second case the second order of accuracy, $o((\Delta t)^2)$, may be achieved with the choice of the numerical flux as follows: 
\begin{eqnarray}
\Delta t F_{g-1/2}&=&-{\rm CFL}\left[E_g-\frac{1-{\rm CFL}}2L(E_{g+1}-E_{g},E_{g}-E_{g-1})\right],\qquad (\nabla\cdot{\bf u})\ge0,\nonumber\\
\Delta t F_{g-1/2}&=&{\rm CFL}\left[E_{g-1}+\frac{1-{\rm CFL}}2L(E_g-E_{g-1},E_{g-1}-E_{g-2})\right],\qquad (\nabla\cdot{\bf u})\le0.\label{eq:mg9}
\end{eqnarray}
where
\begin{equation}
{\rm CFL}=\frac{(\gamma_R-1)\Delta t}{\Delta(\log \varepsilon)}|\nabla\cdot{\bf u}|\ge 0
\end{equation}
is the Courant-Friedrichs-Levi number.
\end{document}