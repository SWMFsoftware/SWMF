Now that you have managed to successfully complete a GITM run you've found yourself with a bunch of output files.  All of the GITM output is in mks units and this data is contained within several files located in the {\tt UA/data} directory, as was previously discussed in Chapter~\ref{quickstart.ch} Section~\ref{post_process.sec}.  You will have found yourself with several {\tt iriOut\_*.dat} files, a {\tt log*.dat} file, and many {\tt .bin} files in whichever formats you specified in SAVEPLOT (see Chapter~\ref{input.ch} Section~\ref{def_out.sec}).  The {\tt iriOut\_*.dat} files are required by the IRI model and not typically used when analyzing the outcome of the GITM run.

The log file provides useful information about the run, such as whether a restart was performed, which physical processes were used, and a list of the universal time, time-step, neutral temperature ranges (T), solar and geomagnetic indices, and the neutral velocity (VV) ranges for each iteration.  This file can be very useful when sharing runs with other users, when revisiting an old run, or merely ensuring that GITM performed as expected.  An example log file is provided below:

\begin{verbatim}
## Inputs from UAM.in
# Resart= F
# Eddy coef:   100.000 Eddy P0:     0.020 Eddy P1:     0.003 Eddy Scaling:     1.000
# Statistical Models Only:  F Apex:  T
# EUV Data:  TFile: 
fismflux.dat                                                                                        
# AMIE: none           
none                                                                                                
# Solar Heating:  T Joule Heating:  T Auroral Heating:  T
# NO Cooling:  T O Cooling:  T
# Conduction:  T Turbulent Conduction:  T Updated Turbulent Conduction:  T
# Pressure Grad:  T Ion Drag:  T Neutral Drag:  T
# Viscosity:  T Coriolis:  T Gravity:  T
# Ion Chemistry:  T Ion Advection:  T Neutral Chemistry:  T
 
#START
   iStep yyyy mm dd hh mm ss  ms      dt min(T) max(T)...
   ...mean(T) min(VV) max(VV) mean(VV) F107 F107A By Bz Vx...
   ...HP HPn HPs SubsolarLon SubsolarLat SubsolarVTEC
       2 2011  9 23  0  0  2 297  2.2979  168.75192  1062.87354...
       ...933.09984 -48.19362    524.93645  1.01910 159.3 127.9 -4.6  0.5 406.9...
       ...11.1 14.4  15.5  3.14145  -0.37655  45.73188
       .
       .
       .
\end{verbatim}

The 3DALL output binary files can contain the following atmospheric quantities:

\begin{itemize}
\item[]{\bf Altitude:} Altitude from the surface of the planet (m)
\item[]{\bf Ar:} Argon density (m$^{-3}$)
\item[]{\bf Ar Mixing Ratio:} Argon mixing ratio
\item[]{\bf CH4 Mixing Ratio:} Methane mixing ratio
\item[]{\bf Conduction:} Heat conduction
\item[]{\bf EuvHeating:} EUV Heating rate
\item[]{\bf H:} Hydrogen density (m$^{-3}$)
\item[]{\bf H!U+!N:} H$^+$ density (m$^{-3}$)
\item[]{\bf H2 Mixing Ratio:} Molecular Hydrogen mixing ratio
\item[]{\bf HCN Mixing Ratio:} Hydrogen Cyanide mixing ratio
\item[]{\bf He:} Helium density (m$^{-3}$)
\item[]{\bf He!U+!N:} He$^+$ density (m$^{-3}$)
\item[]{\bf Heaing Efficiency:} Heating efficiency
\item[]{\bf Heat Balance Total:} Heat balance total
\item[]{\bf Latitude:} Geographic latitude (degrees)
\item[]{\bf Longitude:} Geographic longitude (degrees)
\item[]{\bf N!D2!N:} N$_2$ density (m$^{-3}$)
\item[]{\bf N!D2!U+!N:} N$_2^+$ density (m$^{-3}$) 
\item[]{\bf N!U+!N:} N$^+$ density (m$^{-3}$)
\item[]{\bf N(!U2!ND):} N($^2$D) density (m$^{-3}$)
\item[]{\bf N(!U2!NP):} N($^2$P) density (m$^{-3}$)
\item[]{\bf N(!U4!NS):} N($^4$S) density (m$^{-3}$)
\item[]{\bf N2 Mixing Ratio:} Molecular nitrogen mixing ratio
\item[]{\bf NO:} Nitrious Oxide density (m$^{-3}$)
\item[]{\bf NO!U+!N:} NO$^+$ density (m$^{-3}$)
\item[]{\bf O!D2!N:} O$_2$ density (m$^{-3}$)
\item[]{\bf O!D2!U+!N:} O$_2^+$ density (m$^{-3}$)
\item[]{\bf O(!U1!ND):} O($^1$D) density (m$^{-3}$)
\item[]{\bf O(!U2!ND)!U+!N:} O($^2$D) density (m$^{-3}$)
\item[]{\bf O(!U2!NP)!U+!N:} O($^2$P) density (m$^{-3}$)
\item[]{\bf O(!U3!NP):} O($^3$P) density (m$^{-3}$)
\item[]{\bf O\_4SP\_!U+!N:} O($_4$SP)$^+$ density (m$^{-3}$)
\item[]{\bf RadCooling:} Radiative Cooling rate
\item[]{\bf Rho:} Neutral density (m$^{-3}$)
\item[]{\bf Temperature:} Neutral temperature (K)
\item[]{\bf V!Di!N (east):} Ion velocity towards geographic East (m s$^{-1}$)
\item[]{\bf V!Di!N (north):} Ion velocity towards geographic North (m s$^{-1}$) 
\item[]{\bf V!Di!N (up):} Vertical ion velocity (m s$^{-1}$)
\item[]{\bf V!Dn!N (east):} Neutral velocity towards geographic East (m s$^{-1}$)
\item[]{\bf V!Dn!N (north):} Neutral velocity towards geographic North (m s$^{-1}$)
\item[]{\bf V!Dn!N (up):} Vertical neutral velocity (m s$^{-1}$)
\item[]{\bf V!Dn!N (up,N!D2!N):} Vertical N$_2$ velocity (m s$^{-1}$)
\item[]{\bf V!Dn!N (up,N(!U4!NS)):} Vertical N($^4$S) velocity (m s$^{-1}$)
\item[]{\bf V!Dn!N (up,NO):} Vertical NO velocity (m s$^{-1}$)
\item[]{\bf V!Dn!N (up,O!D2!N):} Vertical O$_2$ velocity (m s$^{-1}$)
\item[]{\bf V!Dn!N (up,O(!U3!NP)):} Vertical O($^3$P) velocity (m s$^{-1}$)
\item[]{\bf e-:} electron density (m$^{-3}$)
\item[]{\bf eTemperature:} electron temperature (K)
\item[]{\bf iTemperature:} ion temperature (K)
\item[]{\bf time:} Universal time
\end{itemize}

There are many routines available to process and analyze the GITM binary files.  The majority of these routines are written in IDL and are available in the {\tt srcIDL} directory within the GITM model directory.  Currently 50 routines have been saved in this directory and more are under development.  Alternatively, python routines are currently being developed and these are located in the {\tt srcPython} directory.  Please not that when using the IDL reader the universal time is read in as epoch seconds from January 1, 1965 00:00 UT, while when using the python reader, the time is imported as a datetime object.

\section{IDL}
\label{idl.sec}

Here is an complete list with some description of the IDL processing and visualization routines currently available.  Please feel free to update this section for other GITM users when you CVS your vetted GITM processing routines.
 
\subsubsection{gitm\_read\_bin}

This is a routine to read a GITM bin file into IDL.  This is great when you want to get a handle on the data and experiment with different visualization methods.

\subsubsection{thermo\_plotsat}

This is the most commonly used routine to plot the 1D GITM results.  It can also be used to plot satellite files and other 1D simulations.  It is relatively straight forward to use, but experimentation can be help.  This is an actual program, so you have to {\tt .run} it.

\subsubsection{thermo\_gui}

This is a someone simplistic graphical user interface code for plotting 3D results.  The filename has to be entered manually in the upper left.  You then have to press the button for loading the file.  Variables appear on the left side, and you can select which one you want to plot.  You then select which of the available planes you would like to look at (lat/lon, lat/alt, or lon/alt) or scroll through the options.  This interface allows you to add wind vectors, plot in polar coordinates, and plot the log of the variable.

\subsubsection{thermo\_batch\_new}

This code will let you look at at 3D files exactly the same way as thermo\_gui, but is all scripted.  There are a few features that this has that thermo\_batch doesn't have:

\begin{enumerate}
%\setlength{\itemsep}{-3in}
	\item You can use wildcards for the file name, so that a list of files can be read.  The postscript file names created for each figure will be differentiated by appending numbers sequentially so that no figures are overwritten.
	\item When plotting a lat/alt plane, you can do a zonal average. 
	\item You can do a global average.
\end{enumerate}

\subsubsection{thermo\_plotter}

All of the above plotting codes will only plot one plot per page.  This code will plot many more than one plot per page.  You can plot multiple variables on the same page, or multiple files with the same variable, or both.

\subsubsection{Other IDL Routines}

Please feel free to provide a description of these routines so that GITM users do not waste their time rewriting code that already exists.

\begin{multicols}{3}
\begin{itemize}
\item{\bf ask}
\item{\bf c\_a\_to\_r}
\item{\bf c\_a\_to\_s}
\item{\bf chopr}
\item{\bf closedevice}
\item{\bf c\_r\_to\_a}
\item{\bf c\_s\_to\_a}
\item{\bf get\_position}
\item{\bf makect}
\item{\bf mklower}
\item{\bf mm}
\item{\bf plotct}
\item{\bf plotdumb}
\item{\bf plotmlt}
\item{\bf pos\_space}
\item{\bf read\_thermosphere\_file}
\item{\bf setdevice}
\item{\bf thermo\_batch}
\item{\bf thermo\_calcforce}
\item{\bf thermo\_champ}
\item{\bf thermo\_compare}
\item{\bf thermo\_compare\_time}
\item{\bf thermo\_convert\_champfiles}
\item{\bf thermo\_guvi}
\item{\bf thermo\_magequator}
\item{\bf thermo\_make\_summary}
\item{\bf thermo\_mkguvisat}
\item{\bf thermo\_mksatsave}
\item{\bf thermo\_mksave}
\item{\bf thermo\_mktec}
\item{\bf thermo\_on2}
\item{\bf thermo\_plotdist}
\item{\bf thermo\_plotlog}
\item{\bf thermo\_plot\_new}
\item{\bf thermo\_plot}
\item{\bf thermo\_plotsat2}
\item{\bf thermo\_plotsat\_constalt\_ON2}
\item{\bf thermo\_plotsat\_constalt}
\item{\bf thermo\_plotvectors}
\item{\bf thermo\_readsat}
\item{\bf thermo\_sigma}
\item{\bf thermo\_superposed}
\item{\bf thermo\_tec}
\item{\bf thermo\_temp}
\item{\bf tostr}
\end{itemize}
\end{multicols}

\section{Python}
\label{python.sec}

This section provides a complete list of the vetted GITM python routines.  These routines require that you use PyBats, a module included in SpacePy.  This is a library developed for space physics applications by the scientists at Los Alamos and can be downloaded for free at: 
{\tt http://spacepy.lanl.gov}

Another library, Basemap, is required for certain plotting routines.  Basemap is a part of the Matplotlib Toolkit and can be installed using Fink, Macports, or downloaded at:
{\tt http://matplotlib.org/basemap/}

If you have questions about these routines or are at the University of Michigan and want to start using Python, Dr. Welling is the man to see.  The source code behind the PyBats GITM routines are located in

\subsubsection{gitm.py}

GITM is a PyBats submodule that handles input and output from GITM.  It can be helpful for those wishing to write their own GITM processing routines but doesn't contain any analysis or visualization routines. 

Once you have downloaded and installed Spacepy, the gitm submodule can be accessed via {\tt import spacepy.pybats.gitm}.  This module contains the following routines:

\begin{itemize}
\item[]{\bf GitmBin: } A routine to load a GITM output bin file.
\item[]{\bf PbData: } The base class for all PyBats data container classes.  Used to hold the GITM data read in using GitmBin
\item[]{\bf dt: } A shortcut for datetime.
\item[]{\bf np: } A shortcut for numpy.
\item[]{\bf dmarray: } A shortcut for data arrays.  Used in many of the data container classes defined in PbData.
\end{itemize}

\subsubsection{gitm\_plot\_rout.py}

Common routines used to format and analyze GITM data.

\begin{itemize}
\item[]{{\bf add\_colorbar:}  Add a color bar to a contour plot.  This routine does not depend on SpacePy.}
\item[]{{\bf center\_polar\_cap:}  Adjust radial coordinates to produce a centered polar plot.  Necessary for the northern hemisphere, where polar plots assume the radial (latitude) coordinates should be centered at zero instead of 90$^\circ$.  This routine does not depend on SpacePy.}
\item[]{{\bf find\_zlimits:}  Find the upper and lower limits for a list of GITM data arrays.}
\item[]{{\bf localtime\_to\_glon:}  Find the longitude at a specified universal time and local time.}
\end{itemize}

\subsubsection{gitm\_3D\_global\_plots.py}

Routines to build and output GITM output variable contour plots over a geographic range.  Several different standard plot formats are available, and routines useful for creating custom figures are also included.

\begin{itemize}
\item[]{{\bf plot\_single\_3D\_image:}  This is a basic visualization routine that creates a filled contour plot of a single output variable from a GITM 3D at a specified altitude or 2D bin file.  The output variable is plotted as a function of latitude and longitude over the entire globe, though the latitude range may be limited.  The output plot may be polar or rectangular and a map of the Earth's continental boundaries may also be included in the output figure.  Sample output of the electron temperature is shown in figure~\ref{gitm_3D_global_plots.fig}~(a) and~(b).}
\item[]{{\bf plot\_single\_nsglobal\_3D\_image:}  A quick way to examine GITM output at both poles.  This routine creates two polar contour plots centered at the geographic northern and southern poles for a single output variable from a GITM 3D at a specified altitude or 2D bin file.  The equatorial and polar latitude boundaries may both be specified, though they cannot change between hemispheres.  The Earth's continental boundaries may also be included in the output figure.  Sample output of the electron temperature is shown in figure~\ref{gitm_3D_global_plots.fig}~(c)}
\item[]{{\bf plot\_global\_3D\_snapshot:}  A snapshot of a single GITM output over the entire globe.  This routine creates two polar contour plots centered at the geographic northern and southern poles and extending to 45$^\circ$ and a single rectangular plot containing the mid- and low-latitudes for a single output variable from a GITM 3D at a specified altitude or 2D bin file.  The Earth's continental boundaries may also be included in the output figure.  Sample output of the electron temperature is shown in figure~\ref{gitm_3D_global_plots.fig}~(d)}
\item[]{{\bf plot\_mult\_3D\_slices:}  This routine creates a single plot containing multiple global contours of a GITM output variable from a 3D or 2D bin file at a list of specified altitudes.  These plots may be either polar or rectangular, with or without continental outlines, and within a specified latitude range.  Sample output of the electron temperature is shown in figure~\ref{gitm_3D_mult_plots.fig}.}
\item[]{{\bf plot\_rectangular\_3D\_global:} This routine plots a single rectangular filled contour for a GITM output variable at a specified altitude index as a function of latitude and longitude.  Title, colorbar, and continental outlines are optional.  A handle to the contour plot is returned to allow the output to be further manipulated depending on what other subplots are included in the output figure.} 
\item[]{{\bf plot\_polar\_3D\_global:} This routine plots a single polar filled contour for a GITM output variable at a specified altitude index as a function of latitude and longitude.  Title, colorbar, and continental outlines are optional.  A handle to the contour plot is returned to allow the output to be further manipulated depending on what other subplots are included in the output figure.  The longitude at the top of the plot may also be specified, this allows one to ensure a specific local time is always located at the top of the dial using a routine like {\bf localtime\_to\_glon}.} 
\end{itemize}

\begin{figure}
\begin{center}
\subfigure[]{
\noindent\includegraphics[width=.45 \textwidth]{Figures/plot_single_3D_image_polar.png}
}
\subfigure[]{
\noindent\includegraphics[width=.45 \textwidth]{Figures/plot_single_3D_image_rect.png}
}
\subfigure[]{
\noindent\includegraphics[width=.45 \textwidth]{Figures/plot_single_nsglobal_3D_image.png}
}
\subfigure[]{
\noindent\includegraphics[width=.45 \textwidth]{Figures/plot_global_3D_snapshot.png}
}
\caption{GITM electron temperature at 456.63 $km$ altitude for: (a) northern latitudes, (b) over the entire globe, (c) over the entire globe, as viewed from the poles, and (d) as a global snapshot.}
\label{gitm_3D_global_plots.fig}
\end{center}
\end{figure}

\begin{figure}
\begin{center}
\subfigure[]{
\noindent\includegraphics[height=.9 \textheight]{Figures/plot_mult_3D_slices_polar.png}
}
\subfigure[]{
\noindent\includegraphics[height=.9 \textheight]{Figures/plot_mult_3D_slices_rect.png}
}
\caption{GITM electron temperature at seven altitude slices for (a) northern latitudes and (b) the entire globe.}
\label{gitm_3D_mult_plots.fig}
\end{center}
\end{figure}

\subsubsection{gitm\_alt\_plots.py}

Routines to build and output GITM output variable linear and contour plots over an altitude range.  Several different standard plot formats are available, and routines useful for creating custom figures are also included.

\begin{itemize}
\item[]{{\bf plot\_single\_alt\_image:}  Creates a single linear or contour altitude plot.}
\item[]{{\bf plot\_mult\_alt\_image:}  Creates a figure with multiple linear or contour altitude plots.}
\item[]{{\bf plot\_alt\_slices:}  Creates a figure with a contour plot showing the altitude dependence of a quantity as a function of latitude or longitude with several linear altitude slices at specified locations.  An example is shown in figure~\ref{gitm_alt_slices.fig}}
\item[]{{\bf plot\_linear\_alt:}  Plots the the linear altitude dependence of a quantity, with altitude on the y-axis.}
\item[]{{\bf plot\_3D\_alt:}  Plots the altitude dependence of a quantity as the function of another spatiotemporal coordinate with the spatiotemporal coordinate on the x-axis, altitude on the y-axis, and the desired quantity as a color contour.}
\end{itemize}

\begin{figure}
\begin{center}
\noindent\includegraphics[width=\textwidth]{Figures/gitm_alt_slice_test_Te.png}
\caption{GITM electron temperature at a constant longitude with six latitude slices.}
\label{gitm_alt_slices.fig}
\end{center}
\end{figure}









