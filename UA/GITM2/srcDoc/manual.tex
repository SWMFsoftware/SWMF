\documentclass[twoside,10pt]{book}
\usepackage{times}
\usepackage{graphicx}
\usepackage{alltt}
\usepackage{amsmath}
\usepackage{epsfig}
\usepackage{fancyheadings}
\usepackage[square]{natbib}

% use these lengths for a more uniform margin
% this format is more pleasing for stapling
\setlength{\oddsidemargin}{-.1 in}
\setlength{\evensidemargin}{0.0 in}
\setlength{\textwidth}{6.5 in}
\setlength{\topmargin}{0 in}
\setlength{\textheight}{8.5 in}

\renewcommand{\deg}{^{\circ}}


\title{GITM User Manual \\ \large Version 2.0}
\author{A.J. Ridley}

\begin{document}

\pagestyle{fancy}
\lhead[\fancyplain{}{\bfseries\thepage}]{\fancyplain{}{\bfseries\rightmark}}
\rhead[\fancyplain{}{\bfseries\leftmark}]{\fancyplain{}{\bfseries\thepage}}
\cfoot{}

%\pagestyle{headings}

\maketitle

\tableofcontents

\clearpage

\chapter{Introduction to GITM}
\input{intro}

\section{Code Outline}

This is an outline of GITM.  To produce this file, go into the {\tt src} 
directory, and type:
\begin{verbatim}
cd ../src
./calling_sequence.pl > outline.tex
mv outline.tex ../srcDoc
cd ../srcDoc
\end{verbatim}

\input{outline.tex}

\chapter{Running GITM}

\section{Quick Start}

\subsection{Extracting the code from a tar file}

Create a new and empty directory, and open the tar file you received,
e.g.:
\begin{verbatim}
mkdir Gitm 
cd Gitm 
mv ../gitm.tgz . 
tar -xvzf gitm.tgz 
\end{verbatim}

\subsection{Checking out the code with CVS}

If CVS (Concurrent Versions System) is available on your computer and
you have an account on the CVS server machine herot.engin.umich.edu,
you can use CVS to install the current or a particular version of the
code. First of all have the following environment variables:
\begin{verbatim}
setenv CVSROOT UserName@herot.engin.umich.edu:/CVS/FRAMEWORK
setenv CVS_RSH ssh
\end{verbatim}
where UserName is your user name on herot. Here it is assumed that you
use csh or tcsh. Also put these settings into your .cshrc file so it
is automatically executed at login. Alternatively, use
\begin{verbatim}
CVSROOT=UserName@herot.engin.umich.edu:/CVS/FRAMEWORK 
export CVSROOT 
CVS_RSH=ssh 
export CVS_RSH
\end{verbatim}
under sh, ksh and bash shells, and also put these commands into your
.bashrc or .profile file so it is automatically executed at login.

Once the CVS environment variables are set, you can download the
current (HEAD) version of the GITM distribution with
\begin{verbatim}
cvs checkout GITM2
\end{verbatim}

If you want a particular version, use
\begin{verbatim}
cvs checkout -r v2_0 GITM2
\end{verbatim}
where v2\_0 is the it tag associated with the version. To download bug
fixes or new features, the
\begin{verbatim}
cvs update 
\end{verbatim}
command can be used. See {\tt man cvs} for more information.

A lot of times, you don't really want the {\tt GITM2} directory to
stay that name, since you might download a couple different version
(maybe one for development and one for runs).  Therefore, typically
you will:
\begin{verbatim}
mv GITM2 GITM2.Development
\end{verbatim}

\subsection{Configuring and Making GITM}

In order to compile GITM, you have to configure it first.  The
configure script is inhereted from the Space Weather Modeling
Framework.  There are two primary reasons you need to do the
configure: (1) put the right {\tt Makefile} in the right place,
specifying the compiler and the version of MPI that you will use to
link the code; (2) put the right MPI header in the right place.  It
also does some things like hard-codes the path of the source code into
the {\tt Makefile}. Some examples are below.

Installing on Nyx:
\begin{verbatim}
./Config.pl -install -compiler=ifortmpif90 -earth
\end{verbatim}

Installing on a Mac with gfortran and openmpi:
\begin{verbatim}
./Config.pl -install -compiler=gfortran -earth
\end{verbatim}

Installing on a computer with gfortran and not using MPI:
\begin{verbatim}
./Config.pl -install -compiler=gfortran -earth -nompi
\end{verbatim}

Sometimes people have a hard time with the ModUtilities.F90 file.  If
you have errors with this file, try (for example):
\begin{verbatim}
./Config.pl -uninstall
./Config.pl -install -compiler=gfortran -earth -noflush
\end{verbatim}

\subsection{Running the Code}

GITM requires a bunch of files to be in the right place in order to
run.  Therefore, it is best to use the makefile to create a run
directory:
\begin{verbatim}
make rundir
mv run myrun
\end{verbatim}
where {\tt myrun} can be whatever you want.  I will use {\tt
myrun} as an example.  You can actually put this directory where ever
you want.  On many system, there is a {\tt nobackup} scratch disk that
you are supposed to use for runs, instead of your home directory.
Therefore, a lot of times, I do the following:
\begin{verbatim}
make rundir
mv run /nobackup/myaccount/gitm/myrun
ln -s /nobackup/myaccount/gitm/myrun .
\end{verbatim}
Then you can treat the run directory just like it is in your home
directory, but it isn't! Once you have created the run directory, you
can run the default simulation, by:
\begin{verbatim}
cd myrun
mpirun -np 4 GITM.exe
\end{verbatim}
This, hopefully should run GITM for Earth for 5 minutes.  If it
doesn't work, then you might have mpi set up incorrectly.  The default
is to allow you to run 4 blocks per processor, and the default {\tt
UAM.in} file is set up for 4 blocks, so you could try just running
GITM without mpi, just to see if it works at all:
\begin{verbatim}
./GITM.exe
\end{verbatim}
If that doesn't work, then it probably didn't compile correctly.
Hopefully, it just worked!

\subsection{Post Processing}

GITM, by default, produces one file per block per output.  If you are
outputting often and you are running with many blocks, you can produce
a huge number of files.  To post process all of these files, simply:
\begin{verbatim}
cd UA
./pGITM
\end{verbatim}
This merges all of the files for one time period, for one file type
into the same file.  You can actually running this while the code is
running, since GITM doesn't use old files.  Well, this isn't actually
true.  Since I just got the APPENDFILE thing working, if you are using
satellites, I won't run this during a run just yet.  I need to test
this feature.  If you are NOT using satellites and NOT using
APPENDFILE, then you are free and clear to use this as often as you
want during a run.  I have run into a case in which GITM was in the
middle of writing a file, and I ran this and it deleted the file that
GITM was writing.  Oops!  This has only happened to me once, though.
I typically run this in a script in which there is a five minute pause
between running it, just to be on the safe side.  I also tend to have
a script running that does an rsync to my computer.  So, the script
runs {\tt pGITM}, then rsyncs the directory (excluding all of the
files that are not processed), and then removing the processed and
rsynced files.  A very simple, but very useful script!


\subsection{The Code Won't Compile!!}
I’m sorry. I tried to make this work on many different platforms, but
sometimes machines are very specific, and it just doesn't work out of
the box. Here are some ideas on how to quickly get this thing
compiling.

\noindent
{\bf Can't find the right Makefile.whatever}

If make does not work, then there is probably a problem with not
finding the FORTRAN 90 compiler. The platform and machine specific
Makefiles are in srcMake. If you type:
\begin{verbatim}
uname 
ls srcMake
\end{verbatim}

If you don't see a file named something like Makefile.uname (where
uname is the output of the uname command), then you will have to build
a proper general Makefile.

You will need a little a little information about your computer, like
what the mpif90 compiler is called and where it is located. Take a
look at srcMake/Makefile.Linux, and try to figure out what all of the
flags are for your system. Then create a srcMake/Makefile.uname with
the correct information in it.

\noindent
{\bf The compiler doesn't recognize flag -x}

You have an operating system that is recognized, but probably a
different compiler. In the srcMake/Makefile.uname file (where uname is
the output of the uname command), there is a line:

OSFLAGS	= -w -dusty

You need to change this line to something more appropriate for your
compiler. Try deleting the flags and compile. If that doesn't work,
you will have to check the man pages of your compiler.




\section{Setting the Grid}

Setting the grid resolution in GITM is not very complicated, but it
does involve some thought.  There are a few variables that control
this.  In {\tt ModSize.f90}, the following variables are defined:

\begin{verbatim}
integer, parameter :: nLons = 9
integer, parameter :: nLats = 9
integer, parameter :: nAlts = 50

integer, parameter :: nBlocksMax = 4
\end{verbatim}

The first three variables (nLons, nLats and nAlts) define the size of
a single block.  In the example above, there are 9 cells in latitude,
9 cells in longitude and 50 cells in altitude.  The final variable
(nBlocksMax) defines the maximum number of blocks you can have on a
single processor.  Typically, most people run with one single block
per processor, so setting this to ``1'' is almost always fine.  In
theory, this can save memory.

\subsection{Running 3D Over the Whole Globe}

Once the number of cells is defined, then the number of blocks in
latitude and longitude need to be defined.  This is done in the {\tt
  UAM.in} file.  For example, this sets the number of blocks to 8 in
the latitude direction and 8 in the longitude direction:
\begin{verbatim}
#GRID
8           lons
8           lats
-90.0       minimum latitude to model
90.0        maximum latitude to model
0.0         minimum longitude to model
0.0         maximum longitude to model
\end{verbatim}
The number of cells in the simulation domain will then be 72 in
longitude, 72 in latitude, and 50 in altitude.  Given that there are
360$\deg$ in longitude and $180\deg$ in latitude, the resolution would
be $360\deg/72 = 5.0\deg$ and $180\deg/72 = 2.5\deg$ in latitude.  If
one block were put on each processor, 64 processors would be required.
This problem fits quite nicely on either four- or eight-core
processors.  On 12-core processors, this would not work very well at
all.  We have started to run simulations of $1\deg$ in latitude and
$5\deg$ in longitude, which can fit nicely on a 12-core processor
machine.  For example, in {\tt ModSize.f90}:
\begin{verbatim}
integer, parameter :: nLons = 9
integer, parameter :: nLats = 15
\end{verbatim}
and in {\tt UAM.in}:
\begin{verbatim}
#GRID
8		lons
12		lats
\end{verbatim}
This can then run on 96 cores, which is nicely divisible by 12.
Essentially, an infinite combination of cells per block and number of
blocks can be utilized.  Typically, the number of blocks in latitude
and longitude are even numbers.

\subsection{Running 3D Over the Part of the Globe}

GITM can be run over part of the globe - both in latitude and in
longitude.  It can be run over a single polar region (by setting
either the minimum or maximum latitude to be greater (or less) than
$\pm 90\deg$).  If this is selected, message passing over the poles is
implemented.  If the pole is not selected, then boundary conditions
have to be set in {\tt set\_horizontal\_bcs.f90}.  By default, a
continuous gradient boundary condition is used on the densities and
temperatures, while a continuous value is used on the velocity.  This
is true in both latitude and longitude.  In longitude, message passing
is implemented all of the time, but the values are over-written by the
boundary conditions if the maximum and minimum longitude are not equal
to each other.

The longitudinal resolution ($\Delta{\phi}$) is set by:
\begin{equation}
\Delta{\phi} = \frac{\phi_{end} - \phi_{start}}{nBlocksLon \times nCellsLon}
\end{equation}
while, the latitudinal resolution ($\Delta{\theta}$) is set by:
\begin{equation}
\Delta{\theta} = \frac{\theta_{end} - \theta_{start}}{nBlocksLat \times nCellsLat}
\end{equation}

\subsection{Running in 1D}

GITM can run in 1D mode, in which the call to advance\_horizontal is
not completed.  This means that GITM runs exactly the same way, but
ignoring all of the horizontal advection terms.  You have to do two
things to make GITM run in 1D.  First, in {\tt ModSize.f90}:
\begin{verbatim}
integer, parameter :: nLons = 1
integer, parameter :: nLats = 1
integer, parameter :: nAlts = 50

integer, parameter :: nBlocksMax = 1
\end{verbatim}
This tells the code that you only want one single latitude and
longitude location.  To specify the exact location, in {\tt UAM.in}:
\begin{verbatim}
#GRID
1           lons
1           lats
41.75       minimum latitude to model
41.75       maximum latitude to model
275.0       minimum longitude to model
275.0       maximum longitude to model
\end{verbatim}
This is pretty close to some place in Michigan.  GITM will model this
exact point for as long as you specify.  One thing to keep in mind
with running in 1D is that the Earth still rotates, so the spot will
have a strong day to night variation in temperature.  In 3D, the winds
decrease some of the variability between day and night, but in 1D,
this doesn't happen.  So, the results are going to be perfect.  But,
1D is great for debugging.

\subsection{Stretching the Grid}

You can stretch the grid in GITM in the latitudinal direction.  It
takes some practice to get the stretching just the way that you might
like.  Here is an example that we typically use for stretching near
the equator for an equatorial electrodynamics run:
\begin{verbatim}
#STRETCH
0.0            ! Equator
0.7            ! Amount of stretch
0.8            ! more control
\end{verbatim}


\chapter{Inputs}

This sets the starting time of the simulation.
Even when you restart, the starttime should be
to the real start time, not the restart time.
\begin{verbatim}
#STARTTIME
iYear    (integer)
iMonth   (integer)
iDay     (integer)
iHour    (integer)
iMinute  (integer)
iSecond  (integer)
\end{verbatim}

This sets the ending time of the simulation.
\begin{verbatim}
#ENDTIME
iYear    (integer)
iMonth   (integer)
iDay     (integer)
iHour    (integer)
iMinute  (integer)
iSecond  (integer)
\end{verbatim}

This will set a time for the code to just pause.
Really, this should never be used.
\begin{verbatim}
#PAUSETIME
iYear iMonth iDay iHour iMinute iSecond
\end{verbatim}

This is typically only specified in a
restart header.  If you specify it in a start UAM.in
it will start the counter to whatever you specify.
\begin{verbatim}
#ISTEP
iStep     (integer)
\end{verbatim}

This sets the maximum CPU time that the code should
run before it starts to write a restart file and end
the simulation.  It is very useful on systems that
have a queueing system and has limited time runs.
Typically, set it for a couple of minutes short of
the max wall clock, since it needs some time to write
the restart files.
\begin{verbatim}
#CPUTIMEMAX
CPUTimeMax    (real)
\end{verbatim}

If you just want GITM to run MSIS and IRI over and
over again, use this.  Then GITM is never actually
run.  You just get MSIS and IRI.
\begin{verbatim}
#STATISTICALMODELSONLY
UseStatisticalModelsOnly    (logical)
DtStatisticalModels         (real)
\end{verbatim}

This is typically only specified in a
restart header.
It sets the offset from the starttime to the
currenttime. Should really only be used with caution.
\begin{verbatim}
#TSIMULATION
tsimulation    (real)
\end{verbatim}

Sets the F10.7 and 81 day average F10.7.  This is
used to set the initial altitude grid, and drive the
lower boundary conditions.
\begin{verbatim}
#F107
f107  (real)
f107a (real - 81 day average of f107)
\end{verbatim}

This specifies the initial conditions and the
lower boundary conditions.  For Earth, we typically
just use MSIS and IRI for initial conditions.
For other planets, the vertical BCs can be set here.
\begin{verbatim}
#INITIAL
UseMSIS        (logical)
UseIRI         (logical)
If UseMSIS is .false. then :
TempMin        (real, bottom temperature)
TempMax        (real, top initial temperature)
TempHeight     (real, Height of the middle of temp gradient)
TempWidth      (real, Width of the temperature gradient)
\end{verbatim}

This says how to use tides.  The first one is using
MSIS with no tides.  The second is using MSIS with
full up tides. The third is using GSWM tides, while
the forth is for experimentation with using WACCM
tides.
\begin{verbatim}
#TIDES
UseMSISOnly        (logical)
UseMSISTides       (logical)
UseGSWMTides       (logical)
UseWACCMTides      (logical)
\end{verbatim}

If you selected to use GSWM tides above, you
can specify which components to use.
\begin{verbatim}
#GSWMCOMP
GSWMdiurnal(1)        (logical)
GSWMdiurnal(2)        (logical)
GSWMsemidiurnal(1)    (logical)
GSWMsemidiurnal(2)    (logical)
\end{verbatim}

This is probably for damping vertical wind
oscillations that can occur in the lower atmosphere.
\begin{verbatim}
#DAMPING
UseDamping        (logical)
\end{verbatim}

I dont know what this is for...
\begin{verbatim}
#GRAVITYWAVE
UseGravityWave        (logical)
\end{verbatim}

This sets the hemispheric power of the aurora.
Typical it ranges from 1-1000, although 20 is a
nominal, quiet time value.
\begin{verbatim}
#HPI
HemisphericPower  (real)
\end{verbatim}

I dont think that GITM actually uses this unless
the Foster electric field model is used.
\begin{verbatim}
#KP
kp  (real)
\end{verbatim}

The CFL condition sets how close to the maximum time
step that GITM will take.  1.0 is the maximum value.
A value of about 0.75 is typical.  If instabilities
form, a lower value is probably needed.
\begin{verbatim}
#CFL
cfl  (real)
\end{verbatim}

This sets the driving conditions for the high-latitude
electric field models.  This is static for the whole
run, though.  It is better to use the MHD\_INDICES
command to have dynamic driving conditions.
\begin{verbatim}
#SOLARWIND
bx  (real)
by  (real)
bz  (real)
vx  (real)
\end{verbatim}

Use this for dynamic IMF and solar wind conditions.
The exact format of the file is discussed further
in the manual.
\begin{verbatim}
#MHD_INDICES
filename  (string)
\end{verbatim}

This is for using Pat Newells aurora (Ovation).
\begin{verbatim}
#NEWELLAURORA
UseNewellAurora   (logical)
UseNewellAveraged (logical)
UseNewellMono (logical)
UseNewellWave (logical)
UseNewellRemoveSpikes (logical)
UseNewellAverage      (logical)
\end{verbatim}

\begin{verbatim}
#AMIEFILES
cAMIEFileNorth  (string)
cAMIEFileSouth  (string)
\end{verbatim}

The limiter is quite important.  It is a value
between 1.0 and 2.0, with 1.0 being more diffuse and
robust, and 2.0 being less diffuse, but less robust.
\begin{verbatim}
#LIMITER
TypeLimiter  (string)
\end{verbatim}

This will set how much information the code screams
at you - set to 0 to get minimal, set to 10 to get
EVERYTHING.  You can also change which processor is
shouting the information - PE 0 is the first one.
If you set the iDebugLevel to 0, you can set the dt
of the reporting.  If you set it to a big value,
you wont get very many outputs.  If you set it to a
tiny value, you will get a LOT of outputs.
UseBarriers will force the code to sync up a LOT.
\begin{verbatim}
#DEBUG
iDebugLevel (integer)
iDebugProc  (integer)
DtReport    (real)
UseBarriers (logical)
\end{verbatim}

\begin{verbatim}
#THERMO
UseSolarHeating   (logical)
UseJouleHeating   (logical)
UseAuroralHeating (logical)
UseNOCooling      (logical)
UseOCooling       (logical)
UseConduction     (logical)
UseTurbulentCond  (logical)
UseUpdatedTurbulentCond  (logical)
EddyScaling  (real)
\end{verbatim}

\begin{verbatim}
#THERMALDIFFUSION
KappaTemp0    (thermal conductivity, real)
\end{verbatim}

\begin{verbatim}
#VERTICALSOURCES
UseEddyInSolver              (logical)
UseNeutralFrictionInSolver   (logical)
MaximumVerticalVelocity      (real)
\end{verbatim}

\begin{verbatim}
#EDDYVELOCITY
UseBoquehoAndBlelly              (logical)
UseEddyCorrection   (logical)
\end{verbatim}

\begin{verbatim}
#WAVEDRAG
UseStressHeating              (logical)
\end{verbatim}

If you use eddy diffusion, you must specify two pressure
levels - under the first, the eddy diffusion is constant.
Between the first and the second, there is a linear drop-off.
Therefore The first pressure must be larger than the second!
\begin{verbatim}
#DIFFUSION
UseDiffusion (logical)
EddyDiffusionCoef (real)
EddyDiffusionPressure0 (real)
EddyDiffusionPressure1 (real)
\end{verbatim}

\begin{verbatim}
#FORCING
UsePressureGradient (logical)
UseIonDrag          (logical)
UseNeutralFriction  (logical)
UseViscosity        (logical)
UseCoriolis         (logical)
UseGravity          (logical)
\end{verbatim}

\begin{verbatim}
#DYNAMO
UseDynamo              (logical)
DynamoHighLatBoundary  (real)
nItersMax              (integer)
MaxResidual            (V,real)
\end{verbatim}

\begin{verbatim}
#IONFORCING
UseExB                 (logical)
UseIonPressureGradient (logical)
UseIonGravity          (logical)
UseNeutralDrag         (logical)
UseDynamo              (logical)
\end{verbatim}

\begin{verbatim}
#DIPOLE
MagneticPoleRotation   (real)
MagneticPoleTilt       (real)
xDipoleCenter          (real)
yDipoleCenter          (real)
zDipoleCenter          (real)
\end{verbatim}

\begin{verbatim}
#APEX
UseApex (logical)
        Sets whether to use a realistic magnetic
        field (T) or a dipole (F)
\end{verbatim}

\begin{verbatim}
#ALTITUDE
AltMin                (real, km)
AltMax                (real, km)
UseStretchedAltitude  (logical)
\end{verbatim}

If LatStart and LatEnd are set to < -90 and
> 90, respectively, then GITM does a whole
sphere.  If not, it models between the two.
If you want to do 1-D, set nLons=1, nLats=1 in
ModSizeGitm.f90, then recompile, then set LatStart
and LonStart to the point on the Globe you want
to model.
\begin{verbatim}
#GRID
nBlocksLon   (integer)
nBlocksLat   (integer)
LatStart     (real)
LatEnd       (real)
LonStart     (real)
LonEnd       (real)
\end{verbatim}

\begin{verbatim}
#STRETCH
ConcentrationLatitude (real, degrees)
StretchingPercentage  (real, 0-1)
StretchingFactor      (real)
Example (no stretching):
#STRETCH
65.0 ! location of minimum grid spacing
0.0	 ! Amount of stretch 0 (none) to 1 (lots)
1.0  
\end{verbatim}

\begin{verbatim}
#TOPOGRAPHY
UseTopography (logical)
\end{verbatim}

\begin{verbatim}
#RESTART
DoRestart (logical)
\end{verbatim}

This will allow you to change the output cadence
of the files for a limited time.  If you have an event
then you can output much more often during that event.
\begin{verbatim}
#PLOTTIMECHANGE
yyyy mm dd hh mm ss ms (start)
yyyy mm dd hh mm ss ms (end)
\end{verbatim}

For satellite files, you can have one single file
per satellite, instead of one for every output.
This makes GITM output significantly less files.
It only works for satellite files now.
\begin{verbatim}
#APPENDFILES
DoAppendFiles    (logical)
\end{verbatim}

This sets the output files.  The most common type
is 3DALL, which outputs all primary state variables.
Types include : 3DALL, 3DNEU, 3DION, 3DTHM, 3DCHM, 
3DUSR, 3DGLO, 2DGEL, 2DMEL, 2DUSR, 1DALL, 1DGLO, 
1DTHM, 1DNEW, 1DCHM, 1DCMS, 1DUSR.
\begin{verbatim}
#SAVEPLOT
DtRestart (real, seconds)
nOutputTypes  (integer)
Outputtype (string, 3D, 2D, ION, NEUTRAL, ...)
DtPlot    (real, seconds)
\end{verbatim}

\begin{verbatim}
#SATELLITES
nSats     (integer - max = ',nMaxSats,')
SatFile1  (string)
DtPlot1   (real, seconds)
etc...
\end{verbatim}

Sets the time for updating the high-latitude
(and low-latitude) electrodynamic drivers, such as
the potential and the aurora.
\begin{verbatim}
#ELECTRODYNAMICS
DtPotential (real, seconds)
DtAurora    (real, seconds)
\end{verbatim}

\begin{verbatim}
#LTERadiation
DtLTERadiation (real)
\end{verbatim}

This is really highly specific.  You dont want this.
\begin{verbatim}
#IONPRECIPITATION
UseIonPrecipitation     (logical)
IonIonizationFilename   (string)
IonHeatingRateFilename  (string)
\end{verbatim}

You really want a log file.  They are very important.
It is output in UA/data/log*.dat.
You can output the log file at whatever frequency
you would like, but if you set dt to some very small
value, you will get an output every iteration, which
is probably a good thing.
\begin{verbatim}
#LOGFILE
DtLogFile   (real, seconds)
\end{verbatim}

This is for a FISM or some other solar spectrum file.
\begin{verbatim}
#EUV_DATA
UseEUVData            (logical)
cEUVFile              (string)
\end{verbatim}




\section{Auxiliary Input Files}


\subsection{IMF and Solar Wind}

This file controls the high-latitude electric field and aurora when
using models that depend on the solar wind and interplanetary magnetic
field (IMF).  It allows for dynamically controlling these quantities.
You can create realistical IMF files or hypothetical ones.  For
realistic IMF files, we typically use CDF files downloaded from the
CDAWEB ftp site, and IDL code that merges the solar wind and IMF files
to create one single file.  The IDL code also propagates the solar
wind and IMF from L1 to 32 Re upstream of the Earth.  You can use the
{\tt DELAY} statement to shift the time more (e.g.,in the example
below, it shifts by an additional 15 minutes).  The IDL code to
process the CDF files is called {\tt cdf\_to\_mhd.pro}.  It requires
both a solar wind file and an IMF file. For example, the IMF file
would be {\tt ac\_h0\_mfi\_20011231\_v04.cdf} and the solar wind file
would be {\tt ac\_h0\_swe\_20011231\_v06.cdf}.  The code assumes that
the data starts at {\tt \#START}, and ends when it encounters an
error.  This can mean that if there is an error in the data somewhere,
the code will only read up to that point.  To validate that the solar
wind and IMF is what you think it is, it is recommended that you use
the IDL code {\tt imf\_plot.pro}.

Here is an example file:
\begin{verbatim}

This file was created by Aaron Ridley to do some
wicked cool science thing.

The format is:
 Year MM DD HH Mi SS mS   Bx  By   Bz     Vx   Vy   Vz    N        T

Year=year
MM = Month
DD = Day
HH = Hour
Mi = Minute
SS = Second
mS = Millisecond
Bx = IMF Bx GSM Component (nT)
By = IMF By GSM Component (nT)
Bz = IMF Bz GSM Component (nT)
Vx = Solar Wind Vx (km/s)
Vy = Solar Wind Vy (km/s)
Vz = Solar Wind Vz (km/s)
N  = Solar Wind Density (/cm3)
T  = Solar Wind Temperature (K)

#DELAY
900.0

#START
 2000  3 20  2 53  0  0  0.0 0.0  2.0 -400.0  0.0  0.0  5.0  50000.0
 2000  3 20  2 54  0  0  0.0 0.0  2.0 -400.0  0.0  0.0  5.0  50000.0
 2000  3 20  2 55  0  0  0.0 0.0  2.0 -400.0  0.0  0.0  5.0  50000.0
 2000  3 20  2 56  0  0  0.0 0.0  2.0 -400.0  0.0  0.0  5.0  50000.0
 2000  3 20  2 57  0  0  0.0 0.0  2.0 -400.0  0.0  0.0  5.0  50000.0
 2000  3 20  2 58  0  0  0.0 0.0  2.0 -400.0  0.0  0.0  5.0  50000.0
 2000  3 20  2 59  0  0  0.0 0.0  2.0 -400.0  0.0  0.0  5.0  50000.0
 2000  3 20  3  0  0  0  0.0 0.0 -2.0 -400.0  0.0  0.0  5.0  50000.0
 2000  3 20  3  1  0  0  0.0 0.0 -2.0 -400.0  0.0  0.0  5.0  50000.0
 2000  3 20  3  2  0  0  0.0 0.0 -2.0 -400.0  0.0  0.0  5.0  50000.0
 2000  3 20  3  3  0  0  0.0 0.0 -2.0 -400.0  0.0  0.0  5.0  50000.0
 2000  3 20  3  4  0  0  0.0 0.0 -2.0 -400.0  0.0  0.0  5.0  50000.0
\end{verbatim}


To actually read in this file, in {\tt UAM.in}, use the command:

\begin{verbatim}
#MHD_INDICES
filename
\end{verbatim}

\subsection{Hemispheric Power}

These files describe the dynamic variation of the auroral power going
into each hemisphere.  Models such as \cite{fuller87} use the
Hemispheric Power to determine which level of the model it should use.
The Hemispheric Power is converted to a Hemispheric Power Index using the 
formula (check):
\begin{equation}
HPI = 2.09log(HP)^{1.0475}
\end{equation}

Example File 1:

\begin{verbatim}
# Prepared by the U.S. Dept. of Commerce, NOAA, Space Environment Center.
# Please send comments and suggestions to sec@sec.noaa.gov 
# 
# Source: NOAA POES (Whatever is aloft)
# Units: gigawatts

# Format:

# The first line of data contains the four-digit year of the data.
# Each following line is formatted as in this example:

# NOAA-12(S)  10031     9.0  4    .914

# Please note that if the first line of data in the file has a
# day-of-year of 365 (or 366) and a HHMM of greater than 2300, 
# that polar pass started at the end of the previous year and
# ended on day-of-year 001 of the current year.

# A7    NOAA POES Satellite number
# A3    (S) or (N) - hemisphere
# I3    Day of year
# I4    UT hour and minute
# F8.1  Estimated Hemispheric Power in gigawatts
# I3    Hemispheric Power Index (activity level)
# F8.3  Normalizing factor

2000
NOAA-15(N)  10023    35.5  7    1.085
NOAA-14(S)  10044    25.3  7     .843
NOAA-15(S)  10114    29.0  7     .676
NOAA-14(N)  10135   108.7 10    1.682
NOAA-15(N)  10204    36.4  7    1.311
.
.
.
\end{verbatim}

The second example shows a file format that is much better.  This
format started in 2007, while all of the files before this time are of
the first example type.

\begin{verbatim}
:Data_list: power_2010.txt
:Created: Sun Jan  2 10:12:58 UTC 2011


# Prepared by the U.S. Dept. of Commerce, NOAA, Space Environment Center.
# Please send comments and suggestions to sec@sec.noaa.gov 
# 
# Source: NOAA POES (Whatever is aloft)
# Units: gigawatts

# Format:

# Each line is formatted as in this example:

# 2006-09-05 00:54:25 NOAA-16 (S)  7  29.67   0.82

# A19   Date and UT at the center of the polar pass as YYYY-MM-DD hh:mm:ss
# 1X    (Space)
# A7    NOAA POES Satellite number
# 1X    (Space)
# A3    (S) or (N) - hemisphere
# I3    Hemispheric Power Index (activity level)
# F7.2  Estimated Hemispheric Power in gigawatts
# F7.2  Normalizing factor

2010-01-01 00:14:37 NOAA-17 (N)  1   1.45   1.16
2010-01-01 00:44:33 NOAA-19 (N)  1   1.45   1.17
.
.
.
\end{verbatim}


\subsection{Solar Irradiance}


More to come here.

\subsection{Satellites}


\begin{verbatim}
#SATELLITES 
2                    nSats 
guvi.2002041623.in 
15.0                 SatDtPlot 
stfd.fpi.in 
60.0                 SatDtPlot

\end{verbatim}

Here is a sample satellite input file:

\begin{verbatim}
year mm dd hh mm ss msec long lat alt
#START
2002 4 16 23 34 25 0 299.16 -2.21 0.00 
2002 4 16 23 34 25 0 293.63 -1.21 0.00 
2002 4 16 23 34 25 0 291.28 -0.75 0.00 
2002 4 16 23 34 25 0 289.83 -0.45 0.00 
2002 4 16 23 34 25 0 288.79 -0.21 0.00 
2002 4 16 23 34 25 0 287.98 -0.01 0.00 
2002 4 16 23 34 25 0 287.32  0.16 0.00 
2002 4 16 23 34 25 0 286.76  0.31 0.00 
2002 4 16 23 34 25 0 286.26  0.46 0.00 
2002 4 16 23 34 25 0 285.81  0.60 0.00 
2002 4 16 23 34 25 0 285.39  0.74 0.00
\end{verbatim}

At this time, GITM ignores the altitude, and just outputs the entire column.



\chapter{Outputs}

Now that you have managed to successfully complete a GITM run you've found yourself with a bunch of output files.  All of the GITM output is in mks units and this data is contained within several files located in the {\tt UA/data} directory, as was previously discussed in Chapter~\ref{quickstart.ch} Section~\ref{post_process.sec}.  You will have found yourself with several {\tt iriOut\_*.dat} files, a {\tt log*.dat} file, and many {\tt .bin} files in whichever formats you specified in SAVEPLOT (see Chapter~\ref{input.ch} Section~\ref{def_out.sec}).  The {\tt iriOut\_*.dat} files are required by the IRI model and not typically used when analyzing the outcome of the GITM run.

The log file provides useful information about the run, such as whether a restart was performed, which physical processes were used, and a list of the universal time, time-step, neutral temperature ranges (T), solar and geomagnetic indices, and the neutral velocity (VV) ranges for each iteration.  This file can be very useful when sharing runs with other users, when revisiting an old run, or merely ensuring that GITM performed as expected.  An example log file is provided below:

\begin{verbatim}
## Inputs from UAM.in
# Resart= F
# Eddy coef:   100.000 Eddy P0:     0.020 Eddy P1:     0.003 Eddy Scaling:     1.000
# Statistical Models Only:  F Apex:  T
# EUV Data:  TFile: 
fismflux.dat                                                                                        
# AMIE: none           
none                                                                                                
# Solar Heating:  T Joule Heating:  T Auroral Heating:  T
# NO Cooling:  T O Cooling:  T
# Conduction:  T Turbulent Conduction:  T Updated Turbulent Conduction:  T
# Pressure Grad:  T Ion Drag:  T Neutral Drag:  T
# Viscosity:  T Coriolis:  T Gravity:  T
# Ion Chemistry:  T Ion Advection:  T Neutral Chemistry:  T
 
#START
   iStep yyyy mm dd hh mm ss  ms      dt min(T) max(T)...
   ...mean(T) min(VV) max(VV) mean(VV) F107 F107A By Bz Vx...
   ...HP HPn HPs
       2 2011  9 23  0  0  2 297  2.2979  168.75192  1062.87354...
       ...933.09984 -48.19362    524.93645  1.01910 159.3 127.9 -4.6  0.5 406.9...
       ...11.1 14.4  15.5
       .
       .
       .
\end{verbatim}

The output bin files can contain the following atmospheric quantities:

\begin{itemize}
\item[]{\bf Altitude:} Altitude from the surface of the planet (m)
\item[]{\bf Ar:} Argon density (m$^{-3}$)
\item[]{\bf Ar Mixing Ratio:} Argon mixing ratio
\item[]{\bf CH4 Mixing Ratio:} Methane mixing ratio
\item[]{\bf Conduction:} Heat conduction
\item[]{\bf EuvHeating:} EUV Heating rate
\item[]{\bf H:} Hydrogen density (m$^{-3}$)
\item[]{\bf H!U+!N:} 
\item[]{\bf H2 Mixing Ratio:} Molecular Hydrogen mixing ratio
\item[]{\bf HCN Mixing Ratio:} Hydrogen Cyanide mixing ratio
\item[]{\bf He:} Helium density (m$^{-3}$)
\item[]{\bf He!U+!N:} 
\item[]{\bf Heaing Efficiency:} Heating efficiency
\item[]{\bf Heat Balance Total:} Heat balance total
\item[]{\bf Latitude:} Geographic latitude
\item[]{\bf Longitude:} Geographic longitude
\item[]{\bf N!D2!N:} 
\item[]{\bf N!D2!U+!N:} 
\item[]{\bf N!U+!N:} 
\item[]{\bf N(!U2!ND):} 
\item[]{\bf N(!U2!NP):} 
\item[]{\bf N(!U4!NS):} 
\item[]{\bf N2 Mixing Ratio:} Molecular nitrogen mixing ratio
\item[]{\bf NO:} Nitrious Oxide density (m$^{-3}$)
\item[]{\bf NO!U+!N:} 
\item[]{\bf O!D2!N:} 
\item[]{\bf O!D2!U+!N:}
\item[]{\bf O(!U1!ND):} 
\item[]{\bf O(!U2!ND)!U+!N:} 
\item[]{\bf O(!U2!NP)!U+!N:} 
\item[]{\bf O(!U3!NP):} 
\item[]{\bf O\_4SP\_!U+!N:} 
\item[]{\bf RadCooling:} Radiative Cooling rate
\item[]{\bf Rho:} Neutral density (m$^{-3}$)
\item[]{\bf Temperature:} Neutral temperature (K)
\item[]{\bf V!Di!N (east):} Ion velocity towards geographic East (m s$^{-1}$)
\item[]{\bf V!Di!N (north):} Ion velocity towards geographic North (m s$^{-1}$) 
\item[]{\bf V!Di!N (up):} Vertical ion velocity (m s$^{-1}$)
\item[]{\bf V!Dn!N (east):} Neutral velocity towards geographic East (m s$^{-1}$)
\item[]{\bf V!Dn!N (north):} Neutral velocity towards geographic North (m s$^{-1}$)
\item[]{\bf V!Dn!N (up):} Vertical neutral velocity (m s$^{-1}$)
\item[]{\bf V!Dn!N (up,N!D2!N):}
\item[]{\bf V!Dn!N (up,N(!U4!NS)):}
\item[]{\bf V!Dn!N (up,NO):}
\item[]{\bf V!Dn!N (up,O!D2!N):}
\item[]{\bf V!Dn!N (up,O(!U3!NP)):}
\item[]{\bf e-:} electron density (m$^{-3}$)
\item[]{\bf eTemperature:} electron temperature (K)
\item[]{\bf iTemperature:} ion temperature (K)
\item[]{\bf time:} Universal Time
\end{itemize}

There are many routines available to process and analyze the GITM bin files.  The majority of these routines are written in IDL and are available in the {\tt srcIDL} directory within the GITM model directory.  Currently 50 routines have been saved in this directory and more are under development.  Alternatively, python routines are currently being developed and these are located in the {\tt srcPython} directory.

\section{IDL}
\label{idl.sec}

Here is an complete list with some description of the IDL processing and visualization routines currently available.  Please feel free to update this section for other GITM users when you CVS your vetted GITM processing routines.
 
\subsubsection{gitm\_read\_bin}

This is a routine to read a GITM bin file into IDL.  This is great when you want to get a handle on the data and experiment with different visualization methods.

\subsubsection{thermo\_plotsat}

This is the most commonly used routine to plot the 1D GITM results.  It can also be used to plot satellite files and other 1D simulations.  It is relatively straight forward to use, but experimentation can be help.  This is an actual program, so you have to {\tt .run} it.

\subsubsection{thermo\_gui}

This is a someone simplistic graphical user interface code for plotting 3D results.  The filename has to be entered manually in the upper left.  You then have to press the button for loading the file.  Variables appear on the left side, and you can select which one you want to plot.  You then select which of the available planes you would like to look at (lat/lon, lat/alt, or lon/alt) or scroll through the options.  This interface allows you to add wind vectors, plot in polar coordinates, and plot the log of the variable.

\subsubsection{thermo\_batch\_new}

This code will let you look at at 3D files exactly the same way as thermo\_gui, but is all scripted.  There are a few features that this has that thermo\_batch doesn't have:

\begin{enumerate}
%\setlength{\itemsep}{-3in}
	\item You can use wildcards for the file name, so that a list of files can be read.  The postscript file names created for each figure will be differentiated by appending numbers sequentially so that no figures are overwritten.
	\item When plotting a lat/alt plane, you can do a zonal average. 
	\item You can do a global average.
\end{enumerate}

\subsubsection{thermo\_plotter}

All of the above plotting codes will only plot one plot per page.  This code will plot many more than one plot per page.  You can plot multiple variables on the same page, or multiple files with the same variable, or both.

\subsubsection{Other IDL Routines}

Please feel free to provide a description of these routines so that GITM users do not waste their time rewriting code that already exists.

\begin{multicols}{3}
\begin{itemize}
\item{\bf ask}
\item{\bf c\_a\_to\_r}
\item{\bf c\_a\_to\_s}
\item{\bf chopr}
\item{\bf closedevice}
\item{\bf c\_r\_to\_a}
\item{\bf c\_s\_to\_a}
\item{\bf get\_position}
\item{\bf makect}
\item{\bf mklower}
\item{\bf mm}
\item{\bf plotct}
\item{\bf plotdumb}
\item{\bf plotmlt}
\item{\bf pos\_space}
\item{\bf read\_thermosphere\_file}
\item{\bf setdevice}
\item{\bf thermo\_batch}
\item{\bf thermo\_calcforce}
\item{\bf thermo\_champ}
\item{\bf thermo\_compare}
\item{\bf thermo\_compare\_time}
\item{\bf thermo\_convert\_champfiles}
\item{\bf thermo\_guvi}
\item{\bf thermo\_magequator}
\item{\bf thermo\_make\_summary}
\item{\bf thermo\_mkguvisat}
\item{\bf thermo\_mksatsave}
\item{\bf thermo\_mksave}
\item{\bf thermo\_mktec}
\item{\bf thermo\_on2}
\item{\bf thermo\_plotdist}
\item{\bf thermo\_plotlog}
\item{\bf thermo\_plot\_new}
\item{\bf thermo\_plot}
\item{\bf thermo\_plotsat2}
\item{\bf thermo\_plotsat\_constalt\_ON2}
\item{\bf thermo\_plotsat\_constalt}
\item{\bf thermo\_plotvectors}
\item{\bf thermo\_readsat}
\item{\bf thermo\_sigma}
\item{\bf thermo\_superposed}
\item{\bf thermo\_tec}
\item{\bf thermo\_temp}
\item{\bf tostr}
\end{itemize}
\end{multicols}

\section{Python}
\label{python.sec}

This section provides a complete list of the vetted GITM python routines.  These routines require that you use PyBats, a module included in SpacePy.  This is a library developed for space physics applications by the scientists at Los Alamos and can be downloaded for free at: 
{\tt http://spacepy.lanl.gov}

If you have questions about these routines or are at the University of Michigan and want to start using Python, Dr. Welling is the man to see.

\subsubsection{gitm}

GITM is a PyBats submodule that handles input and output from GITM.  It can be helpful for those wishing to write their own GITM processing routines but doesn't contain any analysis or visualization routines. 

Once you have downloaded and installed Spacepy, the gitm submodule can be accessed via {\tt import spacepy.pybats.gitm}.  This module contains the following routines:

\begin{itemize}
\item[]{\bf GitmBin: } A routine to load a GITM output bin file.
\item[]{\bf PbData: } The base class for all PyBats data container classes.  Used to hold the GITM data read in using GitmBin
\item[]{\bf dt: } A shortcut for datetime.
\item[]{\bf np: } A shortcut for numpy.
\item[]{\bf dmarray: } A shortcut for data arrays.  Used in many of the data container classes defined in PbData.
\end{itemize}

\subsubsection{gitm\_3d\_test}

This is a basic visualization routine that creates a contour plot of a single output variable from a GITM 3D bin file.  As the name implies, this routine is still under development and user comments are welcome.

\subsubsection{gitm\_diff\_images}

This is a visualization routine that creates a contour plot of the differences between a single GITM 3D output variable at two different times.



\bibliographystyle{authyear}
\bibliography{wholebib}

\end{document}
