In this chapter we will provide some examples of how to actually use the 
PWOM. After reading this chapter you should be familiar enough with the 
use of the model to setup your own simple runs. Three stand-alone examples are 
provided: 
Two for Earth and one for Saturn. The tests will require two separate 
configurations and compilations. One for each planet.

Additionally one example of using the PWOM as the PW component of the SWMF 
is provided. This example assumes that you already have the SWMF installed. 
 

\section{Earth Examples}

\subsection{Configuration and Compilation for the Earth Examples}
The default configuration of the PWOM is for the planet Earth. To check the 
current configuration, type
\begin{verbatim}
  Config.pl
\end{verbatim}
from the PWOM directory. If the PWOM is not currently configured for Earth, 
then type
\begin{verbatim}
  Config.pl -Earth
\end{verbatim}
and the planet configuration will be changed to Earth. After changing the 
configuration to Earth, the code will need to be recompiled. This can be 
accomplished by simply typing
\begin{verbatim}
  make
\end{verbatim}
To carry out the examples in the following subsections, a run directory is 
needed. The creation of a run directory can be accomplished by typing
\begin{verbatim}
  make rundir
\end{verbatim}
This command creates a directory called 'run'. Changing into the newly created 
run directory, we are now prepared to carry out the example simulations

\subsection{Example 1: Single field line simulation and visualization}
In the run directory there is a PARAM.in file that is set up for 4 
field lines. Changing the \#FIELDLINES command from 4 to 1 allows us to 
simulate a single field line. The initial solution for the field line 
is given in 
\begin{verbatim}
  restartIN/restart_iline0001.dat
\end{verbatim}
The location of the field line can be changed by editing the restart file 
above. The second line of said file gives the latitude and longitude of the 
field line foot point. Modifying these numbers will change the location of the 
field line. The code can then be run by typing. 
\begin{verbatim}
  ./PWOM.exe
\end{verbatim}
The result is given in the plots directory and the restart file is given in 
the restartOUT directory.

The results can be visualized using the IDL routines described in the 
BATS-R-US user manual. Simply change into the plots directory and you will see 
several files labeled as
\begin{verbatim}
  plots_iline*.out
\end{verbatim}
Where the star refers to the line number. Entering IDL you can then visualize 
the solution using the 
\begin{verbatim}
  .r getpict
  .r plotfunc
\end{verbatim}
commands as described in the BATS-R-US manual. 
\subsection{Example 2: Multi field line simulation and visualization}
To carry out a run with a multiple field lines, the first thing that 
needs to be done, is to edit the PARAM.in file. Change the \#FIELDLINES 
command 
from 1 to 125. We then need initial solutions for each field line. Type
\begin{verbatim}
  cp ../Scripts/CreateRestart.pl restartIN/
\end{verbatim}
from the run directory. Change directories into the restartIN directory and 
type
\begin{verbatim}
  CreateRestart.pl 
\end{verbatim}
Returning to the PWOM directory, we are now prepared to carry out a multi field 
line simulation. Simply type 
\begin{verbatim}
  ./PWOM.exe
\end{verbatim}
or use mpirun to execute the code on as many processors as you desire (up to 
the number of field lines).

Each individual field line can be visualized as before, but sometimes a 
horizontal slice is desired. To visualize a 2D horizontal slice change into 
the plots directory. Then type 
\begin{verbatim}
  ../Scripts/PostProcessTec.pl
  ../Scripts/CreateMacro.pl
\end{verbatim}
At each command you will be prompted for the altitude slice of interest. 
There should now be 2D Tecplot files and a corresponding Tecplot 
macro file called Macro.mcr. Now just open up Tecplot and run the macro.

\section{Saturn Example}
Return to the PWOM directory. To change the planet configuration to Saturn, 
simply type
\begin{verbatim}
  Config.pl -Saturn
\end{verbatim}
The code must then be recompiled. To do this, type 
\begin{verbatim}
  make
\end{verbatim}

\subsection{Example 3: Single field line simulation}
Move the old run directory out of the way as follows:
\begin{verbatim}
  mv run run_earth
\end{verbatim}
Then create a new run directory by typing
\begin{verbatim}
  make rundir
\end{verbatim}
Then change into the run directory. Everything should now be prepared 
for a Saturn simulation. Carry out the simulation as before. The visualization 
is then the same as before. 
