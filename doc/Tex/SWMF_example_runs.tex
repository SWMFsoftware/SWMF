\subsection{Configuration and Compilation for the Examples}

Select the SC/BATSRUS component version (this will take a few minutes!),
select UA/GITM instead of the default UA/GITM2 version, 
and set the recommended grid sizes for the tests:
\begin{verbatim}
SetSWMF.pl -v=SC/BATSRUS
SetSWMF.pl -v=UA/GITM
SetSWMF.pl -g=UA:9,9,25,2,2 -g=GM:8,8,8,200,40 -g=SC:4,4,4,1000
\end{verbatim}

Compile the main executable code bin/SWMF.exe (this may take an hour!)
and the post processing code bin/PostIDL.exe,
and finally create a run directory and change the directory
\begin{verbatim}
make
make PIDL
make rundir
cd run
\end{verbatim}

\section{Example 1: create SC steady state}

This example involves the SC component only. It demonstrates
how a steady state solar corona can be obtained from
a magnetogram. The convergence to steady state is accelerated by 
a gradual grid refinement, and a gradual application of the
more-and-more accurate numerical schemes. The final state
is a steady state to high accuracy. 

You can use the SWMF with the settings recommended above
but you only need the SC component and
all other component versions can be Empty.

Copy the PARAM.in and LAYOUT.in files
\begin{verbatim}
cp ../test/PARAM.in.SC PARAM.in
cp ../test/LAYOUT.in.SC LAYOUT.in
\end{verbatim}

Check if everything is OK from the main directory.
Define the number of CPU-s you plan to use after the -n= flag,
for example
\begin{verbatim}
cd ..
Scripts/TestParam.pl -n=32
\end{verbatim}

If there are error messages, fix them. 
For example if there are not enough CPU-s for a component change the LAYOUT
Repeat this until Scripts/TestParam.pl runs silently.

Run the code by submitting a job, or interactively
\begin{verbatim}
cd run
mpirun -np 128 SWMF.exe > runlog.np128
\end{verbatim}

After the run finished postprocess the plot files
\begin{verbatim}
cd SC
pIDL
pTEC
cd ..

######################################
# TEST 4: create SC-IH steady state  #
######################################

# This test involves the SC and IH components only. It demonstrates
# how a steady state inner heliosphere solution can be obtained from
# a steady state solar corona solution.
# The convergence to steady state is accelerated by several means.
# First of all the SC component only provides the boundary conditions,
# it only runs in every 100th iteration. Second, the IH grid is built
# up with a gradual grid refinement, and third the 
# more-and-more accurate and expensive numerical schemes are 
# applied in an optimal sequence. The final state
# is a steady state for SC and IH to high accuracy. 

# !!! This test requires the SC restart files created in TEST 3 !!!

# You can use the SWMF with the settings used in the 8 components test, 
# but you only need the SC and IH components,
# all other component versions can be Empty.

# copy in PARAM and LAYOUT files
cp ../test/PARAM.in.SCIH PARAM.in
cp ../test/LAYOUT.in.SCIH LAYOUT.in

# Check if everything is OK from the main directory
# Define the number of CPU-s you plan to use after the -n= flag.
cd ..
Scripts/TestParam.pl -n=128

# If there are error messages, fix them. 
# For example if there are not enough CPU-s for a component change the LAYOUT
# Repeat this until Scripts/TestParam.pl runs silently.

# run the code by submitting a job, or interactively
cd run
mpirun -np 128 SWMF.exe > runlog.np128

# Postprocess the plot files
cd SC
pIDL
pTEC
cd ../IH
pIDL
pTEC
cd ..

########################################
# TEST 5: create GM initial conditions #
########################################

# This test involves the GM component only. It demonstrates
# how a reasonable global magnetosphere can be obtained.
# The upwind boundary conditions are taken from TEST 4, but they
# are intentionally modified to contain a discontinuity.
# This is for demonstration purposes only, to make the subsequent
# tests more dynamic.
# The convergence to steady state is accelerated by 
# a gradual grid refinement, and a gradual application of the
# more-and-more accurate numerical schemes. The final state
# is a reasonable global magnetosphere solution, but it is not 
# a steady state.

# The parameters of the #SOLARWIND command were obtained
# from the result of the SC-IH steady state. 
# Otherwise this test can be run independently, 
# no restart files are read.

# You can use the SWMF with the settings used in the 8 components test, 
# but you only need the GM component,
# all other component versions can be Empty.

# copy in PARAM and LAYOUT files
cp ../test/PARAM.in.GM PARAM.in
cp ../test/LAYOUT.in.GM LAYOUT.in

# Check if everything is OK from the main directory
# Define the number of CPU-s you plan to use after the -n= flag.
cd ..
Scripts/TestParam.pl -n=64

# If there are error messages, fix them. 
# For example if there are not enough CPU-s for a component change the LAYOUT
# Repeat this until Scripts/TestParam.pl runs silently.

# run the code by submitting a job, or interactively
cd run
mpirun -np 64 SWMF.exe > runlog.np64

# Postprocess the plot files
cd GM
pIDL
pTEC
cd ..

#################################################
# TEST 6: create GM-IM-IE-UA initial conditions #
#################################################

# This test involves the GM, IM, IE and UA components. It demonstrates
# how to obtain a reasonable solution for the 
# coupled global/inner magnetosphere, ionosphere and upper atmosphere.
# The initial global magnetosphere solution is taken from TEST 5.
# The convergence to steady state is accelerated by component subcycling.
# The GM, IM and UA components are called at different frequencies.
# This allows the GM, IM and UA components to reach a reasonable solution
# approximately at the same rate.

# !!! This test requires the GM restart files created in TEST 5 !!!

# You can use the SWMF with the settings used in the 8 components test, 
# but you only need the GM, IM, IE and UA components,
# all other component versions can be Empty.

# copy in PARAM and LAYOUT files
cp ../test/PARAM.in.GMIMIEUA PARAM.in
cp ../test/LAYOUT.in.GMIMIEUA LAYOUT.in

# Check if everything is OK from the main directory
# Define the number of CPU-s you plan to use after the -n= flag.
cd ..
Scripts/TestParam.pl -n=128

# If there are error messages, fix them. 
# For example if there are not enough CPU-s for a component change the LAYOUT
# Repeat this until Scripts/TestParam.pl runs silently.

# run the code by submitting a job, or interactively
cd run
mpirun -np 128 SWMF.exe > runlog.np128

# Postprocess the plot files
cd GM
pIDL
pTEC
cd ../IE
pION -r
cd ..

# To make this run useful for the 8 component test 
# rename the restart directories
cd GM
mv restartIN restartIN_orig
mv restartOUT restartIN
mkdir restartOUT
cd ../IM
rmdir restartIN
mv restartOUT restartIN
mkdir restartOUT
cd ../UA
mv RestartOUT RestartIN
mkdir RestartOUT
cd ..

# To be able to start from a 4 hour simulation time
# the GM/restartIN/restart.H file was copied and modified.
# This is not absolutely necessary, because SWMF will overwrite
# the time read by GM.
cp GM/restartIN/restart.H GM/restartIN/restart.H.4hr

# modify simulation time
emacs M/restartIN/restart.H.4hr
-------------------------------------------
...
#TIMESIMULATION
 1.44000000E+04              tSimulation
...
--------------------------------------------

##########################################################
# TEST 7: Run SC-IH-SP for 4 hours in time accurate mode #
##########################################################

# This test involves the Solar Corona, Inner Heliosphere
# and Solar Energetic Particles components.
# The run starts from a steady state of the SC and IH components
# which was obtained in TEST 4.
# At the beginning of this time accurate run a CME is 
# generated in the SC component. This demonstrates the
# use of an Eruptive Event generator.
# The SP component is coupled, and during the 4 hour evolution
# of the CME, substantial particle acceleration is observed. 
# By the end of the run the CME reaches is the boundary 
# between the SC and IH components, and partially enters the IH domain.
# The test demonstrates the coupling between the SC-IH and SP
# components in a challenging simulation.

# This test requires the restart files for SC and IH
# which were generated in TEST 4. Alternatively open
# the tar file:

# !!! on Linux and OSF1 machines use !!!
tar xzf ../test/scih_restart.tgz

# !!! on the IRIX64 and Darwin machines use !!!
gunzip ../test/scih_restart_sgi.tgz
tar xf ../test/scih_restart_sgi.tar

# You can use the SWMF with the settings used in the 8 components test, 
# but you only need the SC, IH and SP components,
# all other component versions can be Empty.

# copy in PARAM and LAYOUT files
cp ../test/PARAM.in.SCIHSP PARAM.in
cp ../test/LAYOUT.in.SCIHSP LAYOUT.in

# Check if everything is OK from the main directory
# Define the number of CPU-s you plan to use after the -n= flag.
cd ..
Scripts/TestParam.pl -n=128

# If there are error messages, fix them. 
# For example if there are not enough CPU-s for a component change the LAYOUT
# Repeat this until Scripts/TestParam.pl runs silently.

# run the code by submitting a job, or interactively
cd run
mpirun -np 128 SWMF.exe > runlog.np128

# Postprocess the plot files
cd SC
pIDL
pTEC
cd ../IH
pIDL
pTEC
cd ..

# To make the results of this test useful for TEST 1
# the restart files have to be renamed
cd SC
mv restartIN restartIN_orig
mv restartOUT restartIN
mkdir restartIN
cd ../IH
mv restartIN restartIN_orig
mv restartOUT restartIN
mkdir restartIN
cd ..

#########################################################
#                                                       #
# NOTE: Test 1 is the official milestone 10 run.  It    #
# will simultate 600 seconds of a fully coupled run.    #
# On select platforms and with enough processors, it    #
# runs faster than real time.                           #
#                                                       #
# The remainder of the tests are used both to test      #
# subsets of components, and to build up the necessary  #
# restart files to execute test 1.  These tests are     #
# described in more detail in the manual.               #
#                                                       #
#########################################################

#########################################################
# TEST 1: restart 8 components at 4 hours physical time #
#########################################################

# This test involves all 8 components. The run starts 
# from a CME eruption which has evolved for 4 hours.
# By this time the CME is reaching the boundary between
# the SC and IH components at 20 solar radii.
# The test demonstrates that the CME passes this boundary.
# The coupling with the SP component shows that particles are
# accelerated, although the 10 minute run does not allow for
# too much acceleration.
#
# The run is set up to contain a discontinuity at the boundary
# between the GM and IH components at 32 Earth radii upstream
# from the Earth. This is an artificially created discontinuity, 
# but it satisfies the laws of physics. There is a contact
# discontinuity in density and temperature, and a rotational
# discontinuity in the Z component of the magnetic field.
# The test demonstrates that the IH state propagates into the GM domain,
# and the discontinuity interacts with the magnetosphere, which
# affects the components GM, IM, RB, IE and UA.

# !!! on Linux and OSF1 machines use !!!
tar xzf ../test/scih_4hr_restart.tgz
tar xzf ../test/gmimieua_restart.tgz

# !!! on the IRIX64 and Darwin machines use !!!
gunzip ../test/scih_4hr_restart_sgi.tgz
gunzip ../test/gmimieua_restart_sgi.tgz
tar xf ../test/scih_4hr_restart_sgi.tar
tar xf ../test/gmimieua_restart_sgi.tar

# copy in PARAM and LAYOUT files
cp ../test/PARAM.in.8comp PARAM.in
cp ../test/LAYOUT.in.8comp LAYOUT.in

# edit LAYOUT.in so that it fits on your machine

# Check if everything is OK from the main directory
# Define the number of CPU-s you plan to use after the -n= flag.
cd ..
Scripts/TestParam.pl -n=128

# If there are error messages, fix them. 
# For example if there are not enough CPU-s for a component change the LAYOUT
# Repeat this until Scripts/TestParam.pl runs silently.

# run the code by submitting a job, or interactively
cd run
mpirun -np 128 SWMF.exe > runlog.np128

# Postprocess the plot files
cd GM
pIDL
pTEC
cd ../IH
pIDL
pTEC
cd ../SC
pIDL
pTEC
cd ../IE
pION -r
cd ..


#################################################################
# TEST 2: restart 8 components from an approximate steady state #
#################################################################

# This test involves all 8 components. The run starts 
# from a steady state of the SC and IH components.
# At the beginning of this time accurate run a CME is 
# generated in the SC component. This demonstrates the
# use of an Eruptive Event generator.
# Although the SP component is coupled, the 10 minute 
# evolution of the CME does not allow particle acceleration.
# Try TEST 7 to study the particle acceleration.
#
# The run is set up to contain a discontinuity at the boundary
# between the GM and IH components at 32 Earth radii upstream
# from the Earth. This is an artificially created discontinuity, 
# but it satisfies the laws of physics. There is a contact
# discontinuity in density and temperature, and a rotational
# discontinuity in the Z component of the magnetic field.
# The test demonstrates that the IH state propagates into the GM domain,
# and the discontinuity interacts with the magnetosphere, which
# affects the components GM, IM, RB, IE and UA.

# !!! on Linux and OSF1 machines use !!!
tar xzf ../test/scih_restart.tgz
tar xzf ../test/gmimieua_restart.tgz

# !!! on the IRIX64 and Darwin machines use !!!
gunzip ../test/scih_restart_sgi.tgz
gunzip ../test/gmimieua_restart_sgi.tgz
tar xf ../test/scih_restart_sgi.tar
tar xf ../test/gmimieua_restart_sgi.tar

# copy in PARAM and LAYOUT files
cp ../test/PARAM.in.8comp.t=0 PARAM.in
cp ../test/LAYOUT.in.8comp LAYOUT.in

# edit LAYOUT.in so that it fits on your machine

# Check if everything is OK from the main directory
# Define the number of CPU-s you plan to use after the -n= flag.
cd ..
Scripts/TestParam.pl -n=128

# If there are error messages, fix them. 
# For example if there are not enough CPU-s for a component change the LAYOUT
# Repeat this until Scripts/TestParam.pl runs silently.

# run the code by submitting a job, or interactively
cd run
mpirun -np 128 SWMF.exe > runlog.np128

# Postprocess the plot files
cd GM
pIDL
pTEC
cd ../IH
pIDL
pTEC
cd ../SC
pIDL
pTEC
cd ../IE
pION -r
cd ..

\end{verbatim}
