%                **** IMPORTANT NOTICE *****
% This LaTeX file has been automatically produced by ProTeX v. 2.1
% Any changes made to this file will likely be lost next time
% this file is regenerated from its source.
 
\parskip        0pt
\parindent      0pt
\baselineskip  11pt
 
%--------------------- SHORT-HAND MACROS ----------------------
\def\bv{\begin{verbatim}}
\def\ev{\end{verbatim}}
\def\be{\begin{equation}}
\def\ee{\end{equation}}
\def\bea{\begin{eqnarray}}
\def\eea{\end{eqnarray}}
\def\bi{\begin{itemize}}
\def\ei{\end{itemize}}
\def\bn{\begin{enumerate}}
\def\en{\end{enumerate}}
\def\bd{\begin{description}}
\def\ed{\end{description}}
\def\({\left (}
\def\){\right )}
\def\[{\left [}
\def\]{\right ]}
\def\<{\left  \langle}
\def\>{\right \rangle}
\def\cI{{\cal I}}
\def\diag{\mathop{\rm diag}}
\def\tr{\mathop{\rm tr}}
%-------------------------------------------------------------
 
%/////////////////////////////////////////////////////////////
\newpage

\markboth{Source File: PM\_wrapper.f90}{Source File: PM\_wrapper.f90}

\subsubsection{ PM\_set\_param - set parameters for the PM module }


\bigskip
{\sf INTERFACE:}
\begin{verbatim} subroutine PM_set_param(CompInfo, TypeAction)
 \end{verbatim}
{\em USES:}
\begin{verbatim}   use CON_comp_info, ONLY: CompInfoType, get, put   ! component info, access
   use CON_coupler,   ONLY: PM_, set_grid_descriptor ! component ID, descriptor
   use ModConst,      ONLY: cPi                      ! Pi = 3.1415...
   use ModIoUnit,     ONLY: STDOUT_                  ! unit for STDOUT
 
   use PM_ModProc, ONLY: PM_iComm, PM_iProc, PM_nProc ! MPI parameters
   use PM_ModMain, ONLY: PM_Dt, PM_TimeAccurate       ! Some variables to set
   use PM_ModSize, ONLY: PM_n, PM_m                   ! Grid size
   use PM_ModIo,   ONLY: PM_iUnit, PM_Prefix  ! I/O unit and prefix for output
 
   implicit none
 \end{verbatim}
{\em INPUT/OUTPUT ARGUMENTS:}
\begin{verbatim}   type(CompInfoType), intent(inout) :: CompInfo   ! Information for this comp
 \end{verbatim}
{\em INPUT ARGUMENTS:}
\begin{verbatim}   character (len=*), intent(in)     :: TypeAction ! What to do
 \end{verbatim}
{\sf LOCAL VARIABLES:}
\begin{verbatim}   character (len=*), parameter :: NameSub = 'PM_set_param'
 
   logical :: DoTest, DoTestMe
 \end{verbatim}
{\sf DESCRIPTION:}\\
 
     This subroutine serves multiple purposes. It is called multiple times
     with various values for the TypeAction argument. The CompInfo argument
     is a derived type that contains basic information about this component.
     The data in this object should be accessed with the get and put methods
     of the CON\_comp\_info class. This subroutine is first called with
     TypeAction='VERSION' so that the version name is provided by PM to CON.
     Next it is called with TypeAction='MPI' and CON provides the MPI 
     communication group parameters to PM. The next actions are 'STDOUT' and
     possibly 'FILEOUT', which set the prefix and the I/O unit for 
     the verbose output of PM. The 'STDOUT' and 'FILEOUT' actions may be 
     called multiple times during the run.
     In each session the physics module reads its input parameters
     (TypeAction='READ') and checks the parameters ('CHECK').
     The action 'READ' may be called multiple times before the 'CHECK' action.
     After the module is initialized for the session (with the 
     PM\_init\_session method), the physics module provides a 
     description of its grid to CON (TypeAction='GRID').
     

%/////////////////////////////////////////////////////////////

\bigskip
{\sf CONTENTS:}
\begin{verbatim}
  ! Set DoTest and DoTesMe logicals. Check for the name of this subroutine
  ! in the StringTest parameter of the #TEST command in PARAM.in.
  call CON_set_do_test(NameSub, DoTest, DoTestMe)

  if(DoTest)write(*,*)NameSub,' called with TypeAction, iProc=',&
       TypeAction, PM_iProc

  select case(TypeAction)
  case('VERSION')
     ! Provide module name and version number for CON
     call put(CompInfo,                          &
          Use        = .true.,                   &
          NameVersion= 'CODE NAME (developers)', &
          Version    = 1.4 )

  case('MPI')
     ! Set the MPI communicator, the processor rank, and number of PE-s
     call get(CompInfo, iComm=PM_iComm, iProc=PM_iProc, PM_nProc=PM_nProc)

  case('STDOUT')
     ! Set PM_iUnit to STDOUT and prefix to 'PMx:' where x is the PE rank
     PM_iUnit=STDOUT_
     write(PM_Prefix,'(a,i1,a)') 'PM', PM_iProc, ':'

  case('FILEOUT')
     ! Set PM_iUnit to the value provided by CON. No need for prefix.
     call get(CompInfo, iUnitOut=PM_iUnit)
     PM_Prefix=''

  case('READ')
     ! Read input parameters for this component
     call read_param

  case('CHECK')
     ! Get some basic parameters from CON
     call get_time(DoTimeAccurateOut = PM_TimeAccurate)

     ! Check interdependent parameters here for consistency

  case('GRID')
     ! Provide grid description for the coupling toolkit.
     ! The grid is a uniform spherical grid divided into two blocks
     ! along the Theta coordinate: northern and southern hemispheres.
     ! Both hemispheres have a PM_n by PM_m grid.
     ! The module can run on 1 or 2 PE-s,
     ! so the processor array is (/0,0/) or (/0,1/)

     call set_grid_descriptor(          &
          PM_,                          &! component index
          nDim        =2,               &! dimensionality of the grid
          nRootBlock_D=(/2,1/),         &! block array for the 2 hemispheres
          nCell_D     =(/PM_n, PM_m/),  &! grid size for each block
          XyzMin_D    =(/0., 0./),      &! min colatitude and longitude indexes
          XyzMax_D    =(/cPi, 2*cPi/),  &! max colatitude and longitude indexes
          TypeCoord   ='SMG',           &! solar magnetic coordinate system
          iProc_A     =(/0,PM_nProc-1/)) ! processor assigment for the blocks

  case default
     call CON_stop(NameSub//': invalid TypeAction='//TypeAction)
  end select

contains
  !========================================================================
  subroutine read_param
    ! Read parameter via ModReadParam
    use ModReadParam, ONLY: &
         read_line, read_command, read_var, i_line_read, lStringLine

    ! String for the name of the command
    character (len=lStringLine) :: NameCommand
    !---------------------------------------------------------------------
    do
       if(.not.read_line() ) EXIT
       if(.not.read_command(NameCommand)) CYCLE

       select case(NameCommand)
       case("#TIMESTEP")
          call read_var('Dt',PM_Dt)
       ! Read other component parameters
       case default
          if(PM_iProc==0) then
             write(*,'(a,i4,a)')NameSub//': ERROR at line ',i_line_read(),&
                  ' invalid command '//trim(NameCommand)
             if(UseStrict)call CON_stop('Correct PARAM.in!')
          end if
       end select
    end do

  end subroutine read_param
 
\end{verbatim}
 
%/////////////////////////////////////////////////////////////
 
\mbox{}\hrulefill\ 
 
\newpage

\markboth{Source File: PM\_wrapper.f90}{Source File: PM\_wrapper.f90}

\subsubsection{ PM\_init\_session - initialize PM at the beginning of the session }


\bigskip
{\sf INTERFACE:}
\begin{verbatim} subroutine PM_init_session(iSession, TimeSimulation)\end{verbatim}
{\em USES:}
\begin{verbatim}   use PM_ModIo, ONLY: PM_iUnit, PM_Prefix ! I/O unit and prefix for output
 
   implicit none
 \end{verbatim}
{\em INPUT ARGUMENTS:}
\begin{verbatim}   integer,  intent(in) :: iSession       ! session number (starting from 1)
   real,     intent(in) :: TimeSimulation ! seconds from start time
 \end{verbatim}
{\sf LOCAL VARIABLES:}
\begin{verbatim}   character(len=*), parameter :: NameSub = 'PM_init_session'
 
   logical :: DoInitialize = .true.\end{verbatim}


%/////////////////////////////////////////////////////////////

\bigskip
{\sf CONTENTS:}
\begin{verbatim}

  ! Initialize module for the first time
  if(DoInitialize)then
     call PM_setup
     DoInitialize = .false.
  end if

  ! Initialization for all sessions comes here

  ! Here is an example of using PM_iUnit and PM_Prefix
  write(PM_iUnit,*) PM_Prefix, NameSub, ' finished for session ', iSession

 
\end{verbatim}
 
%/////////////////////////////////////////////////////////////
 
\mbox{}\hrulefill\ 
 
\newpage

\markboth{Source File: PM\_wrapper.f90}{Source File: PM\_wrapper.f90}

\subsubsection{ PM\_run - do one time step with PM }


\bigskip
{\sf INTERFACE:}
\begin{verbatim} subroutine PM_run(TimeSimulation, TimeSimulationLimit)
 \end{verbatim}
{\em USES:}
\begin{verbatim}   use PM_ModMain, ONLY: PM_Time ! Simulation time of PM
 
   implicit none
 \end{verbatim}
{\em INPUT/OUTPUT ARGUMENTS:}
\begin{verbatim}   real, intent(inout) :: TimeSimulation   ! current time of component
 \end{verbatim}
{\em INPUT ARGUMENTS:}
\begin{verbatim}   real, intent(in) :: TimeSimulationLimit ! simulation time not to be exceeded
 \end{verbatim}
{\sf LOCAL VARIABLES:}
\begin{verbatim}   character(len=*), parameter :: NameSub = 'PM_run'
   real                        :: Dt                 ! Time step\end{verbatim}


%/////////////////////////////////////////////////////////////

\bigskip
{\sf CONTENTS:}
\begin{verbatim}
  ! Check if simulation times agree (e.g. for restart)
  if(abs(PM_Time-TimeSimulation)>0.0001) then
     write(*,*)NameSub,' PM time=',PM_Time,' SWMF time=',TimeSimulation
     call CON_stop(NameSub//' ERROR: PM and SWMF simulation times differ')
  end if

  ! Limit time step
  Dt = min(PM_Dt, TimeSimulationLimit - TimeSimulation)

  ! Call the appropriate subroutine of the physics module to advance by Dt
  call PM_advance(Dt)

  ! Return the new simulation time after the time step
  TimeSimulation = PM_Time

 
\end{verbatim}
 
%/////////////////////////////////////////////////////////////
 
\mbox{}\hrulefill\ 
 
\newpage

\markboth{Source File: PM\_wrapper.f90}{Source File: PM\_wrapper.f90}

\subsubsection{ PM\_save\_restart - save restart files for PM }


\bigskip
{\sf INTERFACE:}
\begin{verbatim} subroutine PM_save_restart(TimeSimulation)
   implicit none\end{verbatim}
{\em INPUT ARGUMENTS:}
\begin{verbatim}   real,     intent(in) :: TimeSimulation   ! seconds from start time\end{verbatim}


%/////////////////////////////////////////////////////////////

\bigskip
{\sf CONTENTS:}
\begin{verbatim}
  ! Call the appropriate subroutine of the physics module
  call PM_save_restart_files(TimeSimulation)
 
\end{verbatim}
 
%/////////////////////////////////////////////////////////////
 
\mbox{}\hrulefill\ 
 
\newpage

\markboth{Source File: PM\_wrapper.f90}{Source File: PM\_wrapper.f90}

\subsubsection{ PM\_finalize - finalize PM at the end of the run }


\bigskip
{\sf INTERFACE:}
\begin{verbatim} subroutine PM_finalize(TimeSimulation)
 
   implicit none
 \end{verbatim}
{\em INPUT ARGUMENTS:}
\begin{verbatim}   real,     intent(in) :: TimeSimulation   ! seconds from start time\end{verbatim}


%/////////////////////////////////////////////////////////////

\bigskip
{\sf CONTENTS:}
\begin{verbatim}
  ! Call the appropriate subroutines of the physics module
  call PM_save_files_final(TimeSimulation)
  call PM_error_report
 
\end{verbatim}

%...............................................................
