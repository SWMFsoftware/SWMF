%^CFG COPYRIGHT UM
\documentclass[twoside,10pt]{report}

\input HEADER

\title{Test Procedures for the Space Weather Modeling Framework}

\author{G\'abor T\'oth\\
  {\it Center for Space Environment Modeling}\\
  {\it The University of Michigan}}

\begin{document}

\pagestyle{fancy}
\lhead[\fancyplain{}{\bfseries\thepage}]{\fancyplain{}{\bfseries\rightmark}}
\rhead[\fancyplain{}{\bfseries\leftmark}]{\fancyplain{}{\bfseries\thepage}}
\cfoot{}
%\pagestyle{headings} % if fancy heading does not work

\maketitle

\tableofcontents

\chapter{Description of Testing Philosophy}

The Space Weather Modeling Framework (SWMF) is a heterogeneous software
comprising of the core of the framework and the various components
modeling the physics domains. Since the components are developed 
independently it is neither possible nor desirable to enforce a
rigorous and comprehensive testing procedure for each component.
On the other hand we must ensure that the core of the SWMF and
the key components developed at the Center for Space Environment
Modeling (CSEM) is well tested, reliable and portable. 
We also established some base line testing for the whole
framework involving all the components to verify that the SWMF
works as expected on different platforms.

\chapter{Description of Unit Testing}

The core of the SWMF consists of three layers:
\begin{itemize}
\item The super structure:  CON/Control, CON/Interface, CON/Stubs
\item The infra structure: CON/Library, CON/Coupler, share/, util/
\end{itemize}
The super structure, as the name suggests, can only be tested
together with the components and the infra structure. 
This complexity is somewhat reduced if the components are replaced
with {\it stubs} which are implemented in the CON/Stubs directory.

\section{Testing with Stub Components}

The SWMF can be compiled with the stub components if the 
\begin{verbatim}
INT_VERSION = Stubs
\end{verbatim}
is selected in {\tt Makefile.def}. With this choice the core
of SWMF can be tested with simplified parameter files,
such as Para

\section{Testing the Registry}

\section{Testing the Library share}

\section{Testing the Timing Utility}

\chapter{Description of Functionality Testing}

\section{Testing the BATSRUS code}

\chapter{Description of Portability Testing}


\end{document}
